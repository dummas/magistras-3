\documentclass[11pt, a4paper]{article}
\usepackage[left=25mm,right=15mm,top=15mm,bottom=15mm]{geometry}
\usepackage[utf8x]{inputenc}
\usepackage[L7x]{fontenc}
\usepackage[lithuanian]{babel}
\usepackage{url}
\usepackage{float}
\usepackage{graphicx}
\usepackage{amsmath}
\usepackage{fancyhdr}
\usepackage{datetime}
\graphicspath{{img/}}
\usepackage{tikz}
\usepackage{gensymb}
\usepackage{subfig}
\usetikzlibrary{arrows}

\begin{document}

  \begin{titlepage}
    \begin{center}
      \textsc{\LARGE Vilniaus Gedimino Technikos universitetas}\\[2mm]
      \textsc{\Large Elektroninių sistemų katedra}\\[70mm]
      \textsc{\Large Objekto padėties nustatymas taikant MEMS jutiklius}\\[60mm]
      \textsc{\Large Juodraštis}\\[60mm]
      \vfill
      {\large Vilnius \\ \the\year}
    \end{center}
  \end{titlepage}

  \tableofcontents

  \newpage

  \section{Įvadas}

  Užduoties tikslas yra sukurti programinį paketą \textit{Cortex-M4} pagrindu veikiančiai įrangai, kuri sugebėtų komunikuoti su \textit{MPU 9250} devynių ašių MEMS jutikliu ir atiduotu gautus duomenis į pagrindinį kompiuterį.
Pagrindinis kompiuteris elgiasi kaip duomenų priėmimo mazgas ir tolimesnis jų skirstytojas.
Kaip tolimesnis mazgas duomenų apdorojimui yra programa, kuri atlieka pirminį duomenų filtravimą ir pateikia gautus duomenis sistemos naudotojui tolimesnei analizei.

Kuriant programine įrangą yra tikslas pritaikyti atviro kodo licencija, kad būtų užkirstas kelias piktybiniam darbo išnaudojimui, bei apsisaugoti nuo bet kokių teisinių ir socialinių ginčų.
Įterptinių sistemų bendruomenėje programinės įrangos atžvilgiu vyrauja atviro kodo principas.
Dėl šios priežasties atliktam darbui norima pritaikyti kuo atviresnę licenciją, kad nebūtų jokių ribojimų kodo panaudojime.
Dėl tos pačios priežasties, labai paprastėja licencijos pritaikymo uždavinys.
Prieš pritaikant licenciją, reikia patikrinti ar naudojamos programinės įrangos licencija leidžia keisti prieš tai buvusią licenciją.
Kuomet yra pritaikyta labai atvira licencija, kuri visiškai neriboja licencijavimo pagrindus, tuomet pakeistai programinei įrangai licencija pritaikyti yra labai paprasta. 

  \section{MEMS jutikliai}

  Šiame skyriuje apžvelgti MEMS pagreičio ir giroskopo tipo jutikliai. 
Aptariama, kokie yra didžiausi jų privalumai ir trūkumai (didelis duomenų triukšmas ir būdai jiems kompensuoti ar mažinti).

\subsection{Giroskopas}

MEMS tipo jutiklis \cite{perlmutter2012high} yra gaminamas staklėmis iš silikono mikro-apdorojimo būdu, turi labai mažai dalių, o jo gamybos kaštai yra gana maži.

MEMS giroskopai yra veikiami \textit{Coriolio efekto}, kuris sako, jog turint koordinačių ašį, sukantis kampiniu greičiu $w$, objektas su mase $m$, kuris juda greičiu $v$, yra veikiamas jėgos

\begin{equation}
    F_c = -2m(w \times v)
\end{equation}

MEMS giroskopai susideda iš vibracinės kilmės elementų, kurie yra naudojami matuojant \textit{Coriolio} efektą. 
Egzistuoja labai daug vibracinių geometrijų, tarp kurių yra vibracinis ratas ir kamertono giroskopai. 
Paprasčiausia geometrija susideda iš vienos masės, kuri skirta vibruoti viena ašimi. 
Kai tik giroskopas yra pasukamas, įvedama antrinė vibracija statmenai pirminei ašiai dėl \textit{Coriolis} jėgos. 
Dėl to, kampinis greitis gali būti apskaičiuojamas, matuojant antrinį apsisukimą. 
Šiuo metu MEMS jutikliai negali pasiekti tokio tikslumo lygio, kokį siūlo optiniai jutikliai, tačiau yra tikimasi, kad ateityje MEMS jutikliai tikslumu nebus prastesni nei optiniai jutikliai.

MEMS jutikliai turi labai daug privalumų, palyginti su kitais jutikliais \cite{titterton2004strapdown}:

\begin{enumerate}
    \item mažas dydis;
    \item mažas svoris;
    \item patvari konstrukcija;
    \item mažos galios naudojimas;
    \item labai greitas paleidimo laikas;
    \item pigi gamyba, esant dideliam mastui;
    \item patikimi;
    \item reikalauja labai mažai priežiūros;
    \item gali būti naudojami sudėtingomis sąlygomis;
\end{enumerate}

\subsubsection{Nuolatinė dedamoji}

Vidutinis kampo pokyčio matavimas, palaikant visišką giroskopo ramybės būseną, yra laikomas nuolatine giroskopo dedamąja. 
Matuojama yra $\degree/h$. 
Nuolatinės dedamosios klaida $\epsilon$ yra apskaičiuojama integruojant ir priklauso nuo laiko $\theta(t) = \epsilon \cdot t$.

Klaida gali būti nustatyta, panaudojus labai ilgo laikotarpio vidutinę vertę, kuomet giroskopas yra paliktas visiškos ramybės būsenos ir nėra veikiamas jokių išorinių jėgų. 
Kai tik nuolatinė dedamoji yra žinoma, labai yra svarbu ją kompensuoti tikro matavimo metu.

\subsubsection{Atsitiktinis kampinis pokytis}

Jutiklio vertės matavimo metu yra galimas triukšmas, sukeliamas terminių ir mechaninių trukdžių, kurių dažnis yra žymiai didesnis už jutiklio vertės matavimo dažnį.
Tokių veiksnių rezultatas yra signalas, kuriame yra balto triukšmo. 
Tai yra paprasčiausia eilė skaičių, kurių vidurkis lygus nuliui.
Nėra jokios koreliacijos ir yra visiškai atsitiktiniai skaičiai. 
Tokiu būdu kiekvienas atsitiktinis skaičius yra tolygiai paskirstytas ir turi baigtinį $\sigma^2$ pasiskirstymą.

Net ir neišvedus formulės (detalus išvedimas yra pateikiamas \cite{woodman2007introduction}) galima teigti, jog triukšmas įveda ,,atsitiktinio vaikščiojimo'' (angl. \textit{random walk}) klaidą integraliniame signale, kuri yra nulinio vidurkio ir kurios variacija didėja nuo laiko pokyčio šaknies.
\begin{equation}
    \sigma_{\theta} (t) = \sigma \cdot \sqrt{ \delta t \cdot t}
\end{equation}
Čia $\sigma$ yra triukšmo signalo pokytis, $t$ yra visas jutiklio įverčio nuskaitymo periodas, o $\delta t$ yra laiko skirtumas tarp nuskaitymų.

Kadangi dominantis įvertis yra kaip triukšmas, lemiantis integralinį signalą, labai dažnai gamintojai nurodo gaminamo jutiklio atsitiktinį kampinį pokytį. Jis yra žymimas $\degree/\sqrt{h}$. 
Pavyzdžiui, jeigu jutiklis yra $0,3\degree / \sqrt{h}$, tai reiškia, jog po vienos valandos standartinės variacijos pozicijos orientacijos klaida sudarys $0,3\degree$, po dviejų valandų $\sqrt{2} * 0,3 = 0.42 \degree$.

\subsection{Nuolatinės dedamosios stabilumas}

Nuolatinė dedamoji MEMS giroskope keičiasi laikui bėgant dėl virpėjimo triukšmo elektronikos ir kituose įtaisuose, kurie potencialiai gali būti paveikti atsitiktinio virpėjimo. 
Virpėjimo triukšmas yra $1/f$ spektro, kurio efektai yra pastebimi elektroniniuose komponentuose, esant žemiems dažniams. 
Esant aukštiems dažniams virpėjimas yra uždengiamas balto triukšmo. 
Nuolatinės dedamosios nestabilumas, kuris kyla dėl virpėjimo, dažniausiai yra modeliuojamas kaip atsitiktinis kampinis pokytis.

Nuolatinės dedamosios stabilumo įvertis nurodo, kiek gali keistis per fiksuotą laiko tarpą. 
Dažniausiai imamas $100~s$ laiko rėžis, kuomet aplinka visiškai nesikeičia. 
Įvertis žymimas kaip $1\sigma$ ir matuojamas $\degree/h$ arba $\degree/s$, kai įranga yra labai netiksli. 
Nuolatinės dedamosios stabilumas yra modeliuojamas atsitiktinio vaikščiojimo procesu. 
Tai galima interpretuoti turint $B_t$ kaip dedamosios vertė laiku $t$, tuomet $1\sigma$ stabilumas $0,01\degree/h$ per 100 sekundžių reiškia, kad dedamoji laiku $t+100$ yra atsitiktinis skaičius, kurio vertė yra tikėtina $B_t$ su $0,01\degree/h$ variacija. 
Po tam tikro laiko, variacija sukuria atsitiktinį vaikščiojimą, kurio nuokrypis didėja per laiko vieneto šaknį. 
Dėl šitos priežasties nuolatinės dedamosios stabilumas yra žymimas kaip dedamosios atsitiktinis vaikščiojimo matas

\begin{equation}
    BRW (\degree / \sqrt{h}) = \frac{BS(\degree/h)}{\sqrt{t(h)}},
\end{equation}
kur $t$ yra nuolatinės dedamosios stabilumo laikas.

Praktikoje yra truputį kitaip. 
Jeigu stabilumas būtų modeliuojamas kaip atsitiktinis vaikščiojimas, tai integracinis įverčio skaičiavimas smarkiai padidintų klaidos įvertį. 
Dėl to, yra nutarta nustatyti rėžius, kuriuose yra nurodomas stabilumas.

\subsubsection{Temperatūros efektai}

Temperatūros svyravimai kyla dėl aplinkos temperatūros nepastovumo ir pačio jutiklio temperatūros nepastovumo. 
Tokie svyravimai natūraliai įtakoja ir nuolatinę dedamąja. 
Jie visiškai nėra nurodomi nuolatinės dedamosios klaidos skaičiavimuose.

Bet koks likutinis nuolatinės dedamosios įterpimas dėl temperatūros pokyčio smarkiai padidins įverčio klaidą ir klaida laikui bėgant didės.
Santykis MEMS jutikliuose tarp temperatūros ir klaidos yra netiesinis.
Dauguma inercinių matavimo sistemų turi savo viduje temperatūros daviklį, todėl matavimo klaidą galima kompensuoti tokiu būdu. 
Kai kurios matavimo sistemos siūlo automatinį klaidos taisymą.

\subsubsection{Kalibravimo klaidos}

Kalibravimo klaidų terminas susieja grupę klaidų šaltinių, kurie susideda iš santykio faktoriaus, lygiavimo ir giroskopų tiesiškumo. 
Tokio tipo klaidų galima pastebėti tik išoriškai veikiant jutiklį ir stebint nuolatinės dedamosios pokytį. 
Tai sukelia integracinio signalo netikslumų, prie kurių prisideda papildomi svyravimai, kurių dydis yra santykis tarp pokyčio ir vykdymo laiko. 
Dažniausiai tokio tipo klaidas galima išmatuoti ir jas kompensuoti.

% Akcelerometras

\subsection{Pagreičio jutiklis}

Mikro-staklių pagaminti silikoniniai pagreičio jutikliai naudoja tokius pačius principus, kaip ir mechaniniai ar kietieji jutikliai. 
Egzistuoja du pagrindiniai MEMS pagreičio jutiklių tipai. 
Pirmas tipas yra mechaniniai pagreičio jutikliai, kurie yra pagaminti iš silikono. 
Jie veikia lygiai tokiais pačiais principais, kaip ir mechaniniai jutikliai. 
Antras jutiklių tipas - kurie matuoja vibracijos pokyčius vibraciniam elemente, ko pasekoje yra stebimi įtampos pokyčiai.

Pagreičio MEMS jutiklių privalumai yra lygiai tokie patys, kaip ir MEMS giroskopinių jutiklių. 
Taip pat, pagrindinis tokių jutiklių minusas prieš kito gaminimo tipo jutiklius - mažas tikslumas. 

\subsubsection{Nuolatinė dedamoji}

Vertės dedamoji, pagreičio matavimo jutiklyje, yra skirtumas tarp matuojamos vertės ir realios vertės, matuojama $m/s^2$.
Pastovios dedamosios klaida $\epsilon$, po dvigubos integracijos, sukuria klaidą, kuri laikui bėgant didėja keturis kartus. 
Sukaupta klaida, priklausomai nuo pozicijos yra

\begin{equation}
    s(t) = \epsilon \cdot \frac{t^2}{2},
\end{equation}
kur $t$ yra integravimo laikas.

Yra galimybių vertinti dedamąją atliekant matavimus su duotu jutikliu labai ilgą laiką, kuriuo neveikia jokia išorinė jėga. 
Deja, visiškai izoliuoti jutiklio nėra galima, kadangi iškarto erdvėje pradeda veikti gravitacija, kuri veikia jutiklį, patiekdama savo nuolatinę dedamąja. 
To pasekoje, labai svarbu yra žinoti įrenginio pozicija žemės atžvilgiu, iš kurios pusės veikia gravitacinė jėga. Praktikoje tai yra pasiekiama naudojant kalibravimo procedūra, kurios metu įrenginys yra pritvirtinamas prie paviršiaus, kurio orientacija gali būti kontroliuojama labai tiksliai.

\subsubsection{Atsitiktinis pagreičio pokytis}

Pagreičio MEMS jutiklio išėjimo matavimai yra veikiami balto triukšmo. 
Kaip jau buvo paminėta giroskopo atsitiktinio kampo pokyčio poskyryje, balto triukšmo integracija sudaro sąlygas variacijai didėti proporcingai $\sqrt{t}$. 
To pasekoje, jutiklio išėjime yra stebimas atsitiktinis verčių vaikščiojimas, kuris vertinamas $m/s/\sqrt{h}$. Praleidžiant standartinės variacijos išvedimą, kuris yra aprašytas \cite{woodman2007introduction}, galima iškarto teigti, kad pagreičio jutiklio baltas triukšmas sukuria antro lygio atsitiktinį verčių vaikščiojimą pozicijoje, su vidurkiu, kuris lygus nuliui ir standartiniu nuokrypiu

\begin{equation}
    \sigma_{s}(t) \approx \sigma \cdot t^{3/2} \cdot \sqrt{\frac{\delta t}{3}},
\end{equation}
kuris didėja proporcingai nuo $t^{3/2}$. 
Čia $t$ yra visas matavimo laikas, $\delta t$ yra skirtumas tarp įverčio matavimo laikų.

\subsubsection{Vibraciniai triukšmai}

MEMS tipo pagreičio jutikliai yra veikiami vibracijos triukšmo, kuris sukelia nuolatinės dedamosios stabilumo klaidą per laiką. 
Tokie nukrypimai yra dažnai modeliuojami kaip nuolatinės dedamosios atsitiktiniai judėjimai, kaip jau buvo aprašyta giroskopo atveju. 
Naudojantis tokiu modeliu, vibracijos sukuria antro lygio atsitiktinio vaikščiojimo triukšmą pagreičiui, proporcingai $t^{3/2}$ ir trečio lygio atsitiktinį vaikščiojimą, kuris yra proporcingas $t^{5/2}$.

\subsubsection{Temperatūros efektai}

Kaip ir giroskopu atžvilgiu, temperatūros pokyčiai įtakoja vibracinius pokyčius, kas sukelia dedamosios nestabilumus išėjimo signale. 
Dedamosios santykis su temperatūra labai priklauso nuo įrangos, tačiau dažniausiai jis yra ne linijinis. 
Bet koks liekamasis nuolatinės dedamosios komponentas sukelia klaida, kuri didėja keturgubai bėgant laikui. 
Pagreičio jutiklio korekcijos dėl temperatūros kai kurie matavimo įrenginiai automatiškai kompensuoja.

\subsubsection{Kalibravimo klaidos}

Kalibravimo klaidos atsiranda kaip nuolatinės dedamosios klaidos. 
Jie pasirodo įverčiuose tik tuomet, kai jutiklis yra veikiamas kažkokio pagreičio jėgos. 
Taip pat verta stebėti gravitaciją, kadangi ji gali sukelti tokias laikinas klaidas ir tuomet, kai jutiklis yra pastovioje pozicijoje ir nėra veikiamas išorinės pagreičio jėgos.



  \section{Signalų filtrai}

  Šiame skyriuje apžvelgti pagrindiniai signalų filtravimo įrankiai, kurie yra plačiai naudojami inercinėse navigacinėse sistemose.

\subsection{\textit{Kalman} filtras}

    \textit{Kalman} filtras \cite{kalman1960new} yra vienas iš labiausiai naudojamų duomenų sujungimo algoritmų informacijos apdorojimo pramonės srityje \cite{faragher2012understanding}. 
    Pats žymiausias \textit{Kalman} filtro panaudojimas jo ankstyvoje stadijoje yra \textit{Apollo} navigaciniam kompiuteryje, kuris nugabeno \textit{Neil Armstrong} iki mėnulio ir svarbiausia -- atgal. 
    Šiais laikais, \textit{Kalman} filtrą galima rasti beveik kiekviename įrenginyje -- mobiliam telefone, kompiuteriniuose žaidimuose.

    \subsubsection{Diskretaus laiko \textit{Kalman} filtras}

    Tarkime, kad turime linijinę diskretaus laiko sistemą

    \begin{equation}
        x_k = F_{k-1}x_{k-1} + G_{k-1}u_{k-1} + w_{k-1}
    \end{equation}
    \begin{equation}
        y_k = H_k x_k + v_k
    \end{equation}

    Proceso triukšmai ${w_k}$ ir ${v_k}$ yra baltasis, nulinio vidurkio, be koreliacijos ir žinomos kovariacijos matricos

    \begin{equation}
    w_k \sim (0, Q_k)
    \end{equation}
    \begin{equation}
    v_k \sim (0, R_k)
    \end{equation}
    \begin{equation}
    E[w_k w_j^T] = Q_k \delta_{k-j}
    \end{equation}
    \begin{equation}
    E[v_k v_j^T] = R_k\delta_{k-j}
    \end{equation}
    \begin{equation}
    E[v_k w_j^T] = 0,
    \end{equation}

    kur $\delta_{k-j}$ yra Kroneckerio delta funkcija, $\delta_{k-j} = 1$, jeigu $k=j$ ir $\delta_{k-j} = 0$, kai $k \neq j$.
    Tikslas yra spėti $x_k$ būsena, remiantis žiniomis apie sistemos dinamikas ir turimu mėginiu, kuris yra triukšmingas ${y_k}$.
    Informacijos kiekis, kuris yra galimas spėjimui skiriasi nuo problemos, kuria yra norima išspręsti.
    Jeigu egzistuoja visi mėginiai, iki laiko $k$, kuriuos galima panaudoti $x_k$ spėjimui, galima suformuoti tolesnį spėjimą, kuri galima pažymėti kaip $\hat{x}_k^{+}$.
    Ženklas $+$ reikia, kad spėjimas yra tolimesnis.
    Vienas iš būdu kaip galima suformuoti tolimesnę būseną yra apskaičiuoti spėjamą $x_k$ vertę, kuri priklauso nuo turimų mėginių iki laiko $k$:

    \begin{equation}
        \hat{x}_k^{+} = E[x_k|y_1,y_2, \dots , y_k]
    \end{equation}

    Jeigu turimi visi matavimai prieš (ir neįtraukiant) laiką $k$ naudojimui, tuomet galima suformuluoti ankstesnį spėjimą, kuris yra žymimas kaip $\hat{x}_k^{-}$.
    Pagalbinis žymėjimas $-$, reiškia ankstesnį spėjimą.
    Vienas iš būdų suformuluoti ankstesnį spėjimą yra apskaičiuoti $x_k$ vertę, kuri priklauso nuo visų turimų matavimų prieš laiką $k$:

    \begin{equation}
        \hat{x}_k^{-} = E[x_k|y_1,y_2,\dots, y_{k-1}]
    \end{equation}

    Labai svarbu pažymėti, jog $\hat{x}^-_k$ ir $\hat{x}_k^{+}$ abu kartu yra spėjimai tokio pačio lygumo; jie abu yra $x_k$ spėjimai.
    Tačiau, $\hat{x}_k^-$ yra $x_k$ spėjimas prieš $y_k$ matavimą, o $\hat{x}_k^+$ yra spėjimas po to, kai $y_k$ matavimas yra atliktas.
    Natūraliai yra tikimąsi, jog $\hat{x}_k^+$ bus geresnis spėjimas už $\hat{x}_k^-$, kadangi tuo metu yra daugiau informacijos skaičiuojant $\hat{x}_k^+$:

    \begin{enumerate}
        \item $\hat{x}_k^-$ = $x_k$ spėjimas prieš mėginio gavimą laiku $k$
        \item $\hat{x}_k^+$ = $x_k$ spėjimas, po mėginio gavimo laiku $k$
    \end{enumerate}

    Jeigu egzistuoja mėginiai po laiko $k$, kuriuos galima naudoti $x_k$ spėjimui, tuomet galima suformuoti glotnesnį spėjimą.
    Vienas iš būdų suformuoti glotnesnį spėjimą yra apskaičiuoti tikėtiną $x_k$ vertę, kuri priklauso nuo visų turimų įverčių

    \begin{equation}
        \hat{x}_{k|k+N} = E[x_k|y_1,y_2,\dots,y_k,\dots, y_{k+N}],
    \end{equation}
    kur $N$ yra teigimas skaičius, kurio vertė priklauso nuo specifinės problemos, kuri yra sprendžiama.
    Jeigu galima rasti geriausią $x_k$ spėjimą daugiau negu vienu žingsniu greičiau už turimus mėginius, galima suformuluoti spėjamą būseną.

    \begin{equation}
        \hat{x}_{k|k-M} = E[x_k|y_1,y_2, \dots, y_{y-M}],
    \end{equation}
    kur $M$ yra teigiamas skaičius, kurio vertė priklauso nuo sprendžiamo uždavinio.

    Tolimesniam žymėjime bus pateiktas $\hat{x}_0^+$ įvertis, kuris parodys pradinį $x_0$ įvertinimą prieš bet kokius turimus mėginius.
    Pirmas mėginys yra gaunamas laiku $k=1$.
    Kadangi neegzistuoja jokie mėginiai, kuriuos galima panaudoti spėti $x_0$, yra logiška suformuoti $\hat{x}^+_0$ kaip pradinės būsenos $x_0$ tikėtiną įvertį:

    \begin{equation}
        \hat{x}_0^+ = E(x_0)
    \end{equation}

    Spėjimo klaidos kovariacijos matricai pažymėti bus panaudotas $P_k$ žymėjimas.
    $P_k^-$ žymi $\hat{x}_k^-$ įvertinimo klaidos kovariaciją, o $P_k^+$ žymi $\hat{x}_k^+$ įvertinimo klaidą.

    \begin{equation}
        P_k^- = E[(x_k - \hat{x}_k^-)(x_k - \hat{x}_k^-)^T]
    \end{equation}
    \begin{equation}
        P_k^+ = E[(x_k - \hat{x}_k^+)(x_k-\hat{x}_k^+)^T]
    \end{equation}

    Po to, kai yra apdorotas įvertis laike $(k-1)$, gaunamas $x_{k-1}$ įvertinimas (žymimas $\hat{x}_{k-1}^+$) ir įverčio kovariacijos matrica (žymima $P_{k-1}^+$).
    Kai ateina laikas $k$, prieš apdorojant įverčiui laike $k$, apskaičiuojame $x_k$ įvertį (žymima $\hat{x}_k^-$) ir įverčio klaidos kovariacijos matricą (žymima $P_k^-$).
    Tuomet yra apdorojamas gautas įvertis laike $k$ ir patobuliname įvertį $x_k$.
    Gautas galutinis įvertis $x_k$ yra žymimas $\hat{x}_k^+$, o jo kovariacijos matrica $P_k^+$.

    Spėjimo procesas pradedamas nuo $\hat{x}_0^+$, geriausiai žinomos pradinės $x_0$ būsenos.
    Turint $\hat{x}_0^+$, kaip apskaičiuoti $\hat{x}_1^-$?
    Reikia prilyginti $\hat{x}_1^- = E(x_1)$.
    Reikia pažymėti, jog $\hat{x}_0^+ = E(x_0)$, ir prisiminti, kad $x$ vidurkis kinta laike

    \begin{equation}
        \bar{x}_k = F_{k-1}\bar{x}_{k-1} + G_{k-1}u_{k-1}
    \end{equation}

    Tokiu atveju gaunama

    \begin{equation}
        \hat{x}_1^- = F_0\hat{x}_0^+ + G_0u_0
    \end{equation}

    Šita lygtis yra specifinė lygtis, kuri rodo kaip gauti $\bar{x}_1^-$ iš $\bar{x}_0^+$. Labiau bendrinant, gaunama lygtis

    \begin{equation}
        \hat{x}_k^- = F_{k-1}\bar{x}_{k-1}^+ + G_{k-1}u{k-1}
    \end{equation}

    Ir jinai yra vadinama $\bar{x}$ laiko atnaujinimo lygtis.
    Nuo laiko $(k-1)^+$ iki laiko $k^-$, būsenos įvertis kinta tokiu pačiu principu, kaip ir būsenos vidurkis.
    Neegzistuoja jokios papildomos informacijos, kuri leistu atnaujinti būseną tarp laiko $(k-1)^+$ ir $k^-$, todėl reikia tiesiog atnaujinti būsenos įvertį, remiantis sistemos dinaminėmis žiniomis.

    Toliau reikia apskaičiuoti laiko atnaujinimo $P$ lygtį.
    Pradedama nuo $P_0^+$, kuris žymi pradinės būsenos įverčio $x_0$ kovariaciją.
    Jeigu labai gerai yra žinoma apie pradinę sistemos būseną, tuomet $P_0^+ = 0$.
    Jeigu visiškai nėra aišku apie pradinę sistemos būseną, tuomet $P_0^+ = \inf$.
    Bendrai, $P_0^+$ parodo pradinės $x_0$ būsenos neužtikrintumą

    \begin{equation}
        P_0^+ = E[(x_0 - \hat{x}_0^+)(x_0 - \hat{x}_0^+)^T]
    \end{equation}

    Turint $P_0^+$, kaip apskaičiuoti $P_1^-$? Reikia prisiminti kaip linijinės diskretaus laiko būsenos kovariacija kinta su laiku ir gaunam

    \begin{equation}
        P_1^- = F_0 P_0^+F_0^T + Q_0
    \end{equation}

    Tai yra specifinė lygtis, o bendresnė lygtis atrodo taip

    \begin{equation}
        P_k^- = F_{k-1}P_{k-1}^+F_{k-1}^T + Q_{k-1},
    \end{equation}
    ir vadinama $P$ laiko atnaujinimo lygtimi.

    Aukščiau yra išvestos lygtis, kurios yra skirtos apskaičiuoti $\hat{x}$ ir $P$.
    Dabar reikalinga išvesti lygtis, kurios skirtos atnaujinti $\hat{x}$ ir $P$ po atlikto sistemos matavimo.
    Turint $\hat{x}_k^-$, kaip apskaičiuoti $\hat{x}_k^+$? Tiek $x_k^-$ ir $x_k^+$ yra $x_k$ įverčiai.
    Vienintelis skirtumas tarp $\hat{x}_k^-$ ir $\hat{x}_k^+$, kad $\hat{x}_k^+$ yra gaunamas turint mėginį $y_k$.
    Remiantis mažiausių kvadratų teorija, mėginys $y_k$ keičia $x$ konstantos spėjimą:

    \begin{equation}
        K_k = P_{k-1}H_k^T(H_kP_{k-1}H_k^T + R_k)^{-1} = P_kH_k^TR_k^-1
    \end{equation}
    \begin{equation}
        \bar{x}_k = \bar{x}_{k-1} + K_k(y_k - H_k\bar{x}_{k-1})
    \end{equation}
    \begin{equation}
        P_k = (I - K_kH_k)P_{k-1},
    \end{equation}
    kur $\bar{x}_{k-1}$ ir $P_{k-1}$ yra įvertinimas ir jo kovariacija prieš $y_k$ matavimą ir $\bar{x}_k$ ir $P_k$ yra įvertinimas ir jo kovariacija po $y_k$ matavimo.

    Atlikę porą pakeitimų, gauname

    \begin{equation}
        K_k = P_{k}^+H_k^TR_k^{-1}
    \end{equation}
    \begin{equation}
        \bar{x}_k^+ = \bar{x}_k^- + K_k(y_k - H_k\bar{x}_k^-)
    \end{equation}
    \begin{equation}
        P_k^+ = (I-K_kH_k)P_{k}^-
    \end{equation}

    Taip yra gaunamos lygtis, kurios yra naudojamos įverčiams $\bar{x}_k$ ir $P_k$ atnaujinti, po gauto mėginio.
    Matrica $K_k$ aukštesnėse lygtyse yra vadinama \textit{Kalman} augimu.

    \subsubsection{Pavyzdys}

    Geriau suprasti turimą teoriją, pateikiamas Niutono sistemos pavyzdys, kurioje nėra jokios triukšmo, su pozicija $r$, greičiu $v$ ir pastoviu pagreičiu $a$.
    Sistema galima apibūdinti kaip

    \begin{equation}
        \begin{bmatrix} \dot{r} \\ \dot{v} \\ \dot{a} \end{bmatrix} =
        \begin{bmatrix} 0 & 1 & 0 \\ 0 & 0 & 1 \\ 0 & 0 & 0 \end{bmatrix}
        \begin{bmatrix} r \\ v \\ a \end{bmatrix}
    \end{equation}
    \begin{equation}
        \dot{x} = Ax
    \end{equation}

    Diskretizuota sistemos versija (su diskretizacijos periodu $T$) yra užrašoma kaip\

    \begin{equation}
        x_{k+1} = Fx_k,
    \end{equation}
    kur $F$ yra duodamas kaip

    \begin{equation}
        F = exp(AT) = I + AT + \frac{(AT)^2}{2!} + \dots =
        \begin{bmatrix}
            0 & T & T^2/2 \\
            0 & 1 & T \\
            0 & 0 & 1
        \end{bmatrix}
    \end{equation}

    \textit{Kalman} filtras tokiai sistemai yra išreiškiamas

    \begin{equation}
        \bar{x}_k^- = F\bar{x}_{k-1}^+
    \end{equation}
    \begin{equation}
        P_k^- = FP_{k-1}^+ F^T + Q_{k-1} = FP_{k-1}^+ F^T
    \end{equation}

    Ką galima pastebėti, kad kovariacijos įverčio klaidos matrica didėja tarp laiko $(k-1)^+$ (laikas $(k-1)$, po kurio gautas įvertis yra apdorotas) ir $k^-$ (laikas $k$, prieš apdorojant gautą įvertį).
    Kadangi nėra gaunamas joks mėginys tarp laikų $(k-1)^+$ ir $k^-$, yra daug logikos, jog duota kovariacijos matrica didėja.
    Dabar galima įvesti triukšmą, kuris būtų mėginio gavimo triukšmas $\sigma^2$

    \begin{equation}
        y_k = H_kx_k + v_k = \begin{bmatrix} 1 & 0 & 0 \end{bmatrix} x_k + v_k
    \end{equation}
    \begin{equation}
        v_k ~ (0, R_k)
    \end{equation}
    \begin{equation}
        R_k = \sigma^2
    \end{equation}

    \textit{Kalman} augimas gaunamas kaip

    \begin{equation}
        K_k = P_k^-H_k^T(H_kP_k^-H_k^T+R_k)^{-1}
    \end{equation}

    Perrašius $P_k$ kaip trijų elementų vektorių

    \begin{equation}
        K_k = \begin{bmatrix} P_{k,11}^- \\ P_{k,12}^- \\ P_{k,13}^- \end{bmatrix} \frac{1}{P_{k,11}^- + \sigma^2}
    \end{equation}

    Tolimesnė kovariacija gaunama

    \begin{equation}
        P_k^+ = P_k^- - K_kH_kP_k^-
    \end{equation}

    Kai yra gaunamas naujas matavimas, yra natūraliai tikimąsi, jog būsenos spėjimas bus tikslesnis.
    Tai reiškia, jog kovariacija mažės, ką ir rodo lygtis.

    \subsection{Ištiesintas \textit{Kalman} filtras}

    Šiame skyriuje parodoma kaip ištiesinti ne tiesinę sistemą ir tuomet panaudoti \textit{Kalman} filtro teoriją įvertinant skirtumus tarp būsenos ir nominalios būsenos vertės.
    Toliau tai suteiks ne tiesinės sistemos būsenos įvertinimą.
    Ištiesintas \textit{Kalman} filtras yra išvedamas iš nuolatinio laiko požiūrio, o diskretaus ar hibridinio laiko išvedimai yra analogiški.

    Tarkime, egzistuoja toks ne tiesinės sistemos modelis:

    \begin{equation}
        \dot{x} = f(x,u,w,t)
    \end{equation}
    \begin{equation}
        y = h(x,v,t)
    \end{equation}
    \begin{equation}
        w ~ (0,Q)
    \end{equation}
    \begin{equation}
        v ~ (o,R)
    \end{equation}

    Sistemos lygtis $f(\cdot)$ ir matavimo lygtis $h(\cdot)$ yra ne tiesinės funkcijos.
    Bus naudojama Tailoro eilutė išplėsti šitas lygtis aplink nominalią kontrolę $u_0$ ir nominalią būseną $x_0$ su $y_0$ išėjimu, bei triukšmu $w_0$ir $v_0$.
    Šios nominalios vertės yra paremtos ankstesniu spėjimu apie sistemos trajektoriją.
    Pavyzdžiui, jeigu sistemos lygtis pateikia lėktuvo dinamikas, tuomet nominali kontrolė, būsena ir išėjimas gali būti suplanuoto skrydžio trajektorija.
    Reali skrydžio trajektorija skirsis nuo šios nominalios trajektorijos dėl modeliavimo klaidos, trikdymų ir kitų nenumatytų efektų.
    Tačiau reali trajektorija turi būti arti nominalios trajektorijos, tokiu atveju Tailoro eilutės tiesinimas turi būti labai arti.
    Tailoro eilutės išskleidimas suteikia

    \begin{equation}
        \begin{aligned}
            \dot{x} &\approx f(x_0, u_0, w_0, t) + \frac{\delta f}{\delta x}\Bigr|_{0} (x-x_0) + \frac{\delta f}{\delta u}\Bigr|_{0} (u-u_0) + \frac{\delta f}{\delta w}\Bigr|_{0} (w-w_0) \\
            &= f(x_0, u_0, w_0, t) + A\Delta x + B \Delta u + L \Delta w
        \end{aligned}
    \end{equation}
    \begin{equation}
        \begin{aligned}
            y &\approx h(x_0, v_0, t) + \frac{\delta h}{\delta x}\Bigr|_{0}(x-x_0) + \frac{\delta h}{\delta v}\Bigr|_{0}(v-v_0) \\
            &= h(x_0, v_0, t) + C \Delta x + M \Delta v
        \end{aligned}
    \end{equation}

    Šie dalinių matricų A, B, C, L ir M išvestinės yra išskiriamos iš aukštesnių lygčių.
    Indeksas "0" rodo, kad dalinę išvestinę reikia atlikti esant nominaliai kontrolei, būsenai, išėjimui ir triukšmams.

    Galima padaryti prielaida, kad nominalaus triukšmo įverčiai yra abu lygus 0 šiuo atveju.
    Kadangi jie abu yra lygus 0, tuomet $\delta w(t) = w(t)$ ir $\delta v(t) = v(t)$.
    Toliau galima teigti, kad kontrolė $u(t)$ yra labai gerai žinoma.
    Bendru atveju, tai yra labai gera prielaida, kadangi kontrolės įėjimas $u(t)$ yra pateiktas sistemos kontrolės, kuri neturi turėti jokių abejonių įverčiuose.
    Tai reiškia, kad $u_0(t) = u(t)$ ir $\delta u(t) = 0$.
    Tačiau, realybėje gali būti neužtikrintumas kontrolės išėjime, kadangi jos yra sujungtos su kažkokiu įtaisu, kuris turi savo vidurkį ir triukšmą.
    Dabar galima aprašyti nominalios sistemos trajektoriją kaip

    \begin{equation}
        \label{eq:nominal_sistem_trajektory}
        \begin{aligned}
            \dot{x}_0 &= f(x_0, u_0, w_0, t) \\
            y_0 &= h(x_0, v_0, t)
        \end{aligned}
    \end{equation}

    Aprašytas nukrypimas nuo tikros sistemos būsenos išvestinę iš nominalios būsenos išvestinės ir nukrypimas nuo realaus mėginio iš nominalaus mėginio

    \begin{equation}
        \Delta \dot{x} = \dot{x} - \dot{x}_0 \\
        \Delta y = y - y_0
    \end{equation}

    Su šitais apibrėžimais, lygtis tampa

    \begin{equation}
        \Delta \dot{x} = A \Delta x + Lw = A \Delta x \tilde{w}
    \end{equation}
    \begin{equation}
        \tilde{w} \sim (0, \tilde{Q}), \tilde{Q} = LQL^T
    \end{equation}
    \begin{equation}
        \Delta y = C \Delta x + Mv = C \Delta x + \tilde{v}
    \end{equation}
    \begin{equation}
        \tilde{v} \sim (O, \tilde{R}), \tilde{R} = MRM^T
    \end{equation}

    Ši lygtis yra tiesinė sistema su $\Delta x$ būsena ir $\Delta y$ mėginiu.
    Tokiu atveju galima panaudoti \textit{Kalman} filtrą spėti $\Delta x$.
    Filtro įėjimas susideda iš $\Delta y$, kuris yra skirtumas tarp realaus matavimo $y$ ir nominalaus matavimo $y_0$.
    \textit{Kalman} filtro išėjimas $\Delta x$ yra įvertinimas skirtumo tarp realios būsenos $x$ ir nominalios būsenos $x_0$.
    Filtro lygtis ištiesintai sistemai yra

    \begin{equation}
        \label{eq:linerialized_kalman_filter}
        \begin{aligned}
            \Delta \hat{x} (0) &= 0\\
            P(0) &= E[(\Delta x(0) - \Delta \hat{x}(0))(\Delta{x}(0) - \Delta \hat{x}(0))^T]\\
            \Delta \dot{\hat{x}} &= A \Delta \hat{x} \\
            K &= PC^T\tilde{R}^{-1}\\
            \dot{P} &= AP + PA^T + \tilde{Q} - PC^T\tilde{R}^{-1}CP\\
            \hat{x} &= x_0 + \Delta \hat{x}
        \end{aligned}
    \end{equation}

    \textit{Kalman} filtrui, $P$ yra kovariacijos matricos klaida.
    Ištiesintas \textit{Kalman} filtras nėra tikras filtras, dėl klaidų, kurias suteikia ištiesinimas.
    Tačiau, jeigu ištiesinimo klaidos yra mažos, $P$ turėtų būti apie spėjimo klaidos kovariacijos įvertį.


\subsection{Išplėstasis \textit{Kalman} filtras}

    Ankstesniam skyriuje buvo aptartas tiesinis \textit{Kalman} filtras, kuris yra naudojamas spėti ne tiesinės sistemos būseną.
    Išvedimas buvo paremtas ne tiesinės sistemos tiesinimu aplink nominalios būsenos trajektorijos pokytį.
    Klausimas kyla, kaip gi reikia nustatyti tą nominalę būsenos trajektoriją?
    Kai kuriais atvejais, tai nėra labai paprastas uždavinys.
    Tačiau, kadangi \textit{Kalman} filtras spėja sistemos būseną, galima panaudoti patį \textit{Kalman} filtrą, kaip tą sistemos trajektorijos nominalę.
    Tai yra šioks toks savikėlimo metodas.
    Tiesinam ne tiesinę sistemą aplink \textit{Kalman} filtro būsenos spėjimą, o \textit{Kalman} filtro įvertinimas paremtas tiesine sistema.
    Tai yra išplėstojo \textit{Kalman} filtro idėja, kuri buvo pasiūlyta Stanley Schmidt.
    Autorius norėjo pritaikyti \textit{Kalman} filtrą ne tiesinėms erdvėlaivio navigacijos problemoms spręsti.

\subsubsection{Nepertraukiamo laiko išplėstasis \textit{Kalman} filtras}

        Sujungiant \ref{eq:nominal_sistem_trajektory} ir \ref{eq:linerialized_kalman_filter} lygtis, gaunama

        \begin{equation}
            \dot{x}_0 + \delta \dot{\hat{x}} = f(x_0, u_0, w_0, t) + A \Delta \hat{x} + K[y-y_0 -C(\hat{x} - x_0)]
        \end{equation}

        Toliau reikia pasirinkti tokį $x_0(t)=\hat{x}(t)$, kad $\Delta\hat{x}(t) = 0$ ir $\Delta\hat{\bar{x}}(t) = 0$.
        Kitais žodžiais, tiesinimo trajektorija $x_0(t)$ yra lygi ištiesinto \textit{Kalman} filtro įverčiui $\hat{x}(t)$.
        Tuomet nominalaus įverčio išraiška tampa

        \begin{equation}
            y_0 = h(x_0, v_0, t) = h(\hat{x}, v_0, t)
        \end{equation}

        Tai yra ekvivalentu ištiesintam \textit{Kalman} filtrui, tačiau čia pasirinkta $x_0 = \hat{x}$ ir yra perskirstytos lygtis, kad tiesiogiai gauti $\hat{x}$.
        \textit{Kalman} padidėjimas yra toks pats, kaip ir ankščiau.
        Tačiau šita formuluotė naudoja mėginį $y$ tiesiogiai, ir išėjime būsenos įvertis $\hat{x}$ irgi yra pateikiamas tiesiogiai.
        Tai yra dažnai žymima kaip išplėstas Kalman-Bucy filtras, kadangi Ričardas Bucy bendradarbiavo su Rudolfu Kalman pirmoje publikacijoje apie nepertraukiamo laiko \textit{Kalman} filtrą.
        Nepertraukiamo laiko išplėstas \textit{Kalman} filtras gali būti suvestas į tokius punktus

        \begin{enumerate}
            \item Sistemos lygtis yra tokios
            \begin{equation}
                \begin{aligned}
                \bar{x} &= f(x, u, w, t) \\
                y &= h(x, v, t) \\
                w &\sim (0,Q) \\
                v &\sim (0, R)
                \end{aligned}
            \end{equation}
            \item Apskaičiuoti duotas dalines išvestinių matricas kaip dabartinės būsenos įvertinimus
            \begin{equation}
                \begin{aligned}
                    A &= \frac{\delta f}{\delta x}\Bigr|_{\hat{x}} \\
                    L &= \frac{\delta f}{\delta w}\Bigr|_{\hat{x}} \\
                    C &= \frac{\delta h}{\delta x}\Bigr|_{\hat{x}} \\
                    M &= \frac{\delta h}{\delta v}\Bigr|_{\hat{x}}
                \end{aligned}
            \end{equation}
            \item Apskaičiuoti duotas matricas
            \begin{equation}
                \label{eq:ekf_q_r}
                \begin{aligned}
                    \tilde{Q} &= LQL^T \\
                    \tilde{R} &= MRM^T
                \end{aligned}
            \end{equation}
            \item Išspręsti \textit{Kalman} filtro lygtis
            \begin{equation}
                \label{eq:ekf}
                \begin{aligned}
                    \hat{x}(0) &= E[x(0)]  \\
                    P(0) &= E[ (x(0) - \hat{x}(0))(x(0) - \hat{x}(0))^T ] \\
                    \dot{\hat{x}} &= f(\hat{x}, y, w_0, t) + K[y - h(\hat{x}, v_0, t)] \\
                    K &= PC^T\tilde{R}^{-1} \\
                    \dot{P} &= AP + PA^T + \tilde{Q} - PC^T\tilde{R}^{-1}CP
                \end{aligned}
            \end{equation}
        \end{enumerate}

\subsection{Sekamasis \textit{Kalman} filtras}

%Unscented Kalman filter

Kaip buvo aptarta ankščiau, išplėstasis \textit{Kalman} filtras yra plačiausiai naudojamas būsenos įvertinimo algoritmas netiesinėms sistemoms.
Tačiau, šis filtras gali būti sunkus derinti ir dažnai gaunamas rezultatas yra nepatikimas, jeigu sistema blogai priima tiesinimą.
Tai yra dėl to, kad filtras labai stipriai remiasi tiesinimu skleisti būsenos vidurkį ir kovariaciją.
Kaip alternatyva yra sekamasis \textit{Kalman} filtras, kuris sumažina tiesinimo klaidas.

Sekamąja transformacija yra paremta dviems principais.
Pirma, yra lengva atlikti ne tiesinę transformaciją ant vieno taško (skirtingai nuo viso dokumento transformaciją).
Antra, nėra labai sudėtinga rasti būsenos taškų rinkinį erdvėje, kurie apytiksliai parodytų realią dokumento būseną.

Remiantis šiomis idėjomis, galima teigti, kad yra žinomas vektoriaus $x$ vidurkis $\hat{x}$ ir kovariacija $P$.
Tuomet reikia rasti deterministinį vektorių rinkinį, kurie vadinasi sigma taškai, kuriuos apibūdina vidurkis ir kovariacija, kurie savo ruoštu yra lygus $\hat{x}$ ir $P$.
Sekantis žingsnis yra pritaikyti ne tiesinę funkciją $y = h(x)$ kiekvienam deterministiniam vektoriui ir gauti transformuotus vektorius.
Apibūdinantis vidurkis ir kovariacija duos bendrą supratimą apie tikrąjį $y$ vidurkį ir kovariaciją.
Tai yra raktas sekamajai transformacijai.

Kaip pavyzdį galima pateikti $x$, kuris yra $n~x~1$ vektorius, kuris yra transformuotas ne tiesinės funkcijos $y = h(x)$.
Imami $2n$ $x^{(i)}$ sigma taškai:

\begin{equation}
    \begin{aligned}
    x^{(i)} &= \tilde{x} + \tilde{x}^{(i)}, i = 1, \cdot, 2n \\
    \tilde{x}^{(i)} &= (\sqrt{nP})^T_i, i = 1,\cdot,n\\
    \tilde{x}^{(n+i)} &= -(\sqrt{nP})^T_i, i = 1, \cdot, n,
    \end{aligned}
\end{equation}
kur $\sqrt{nP}$ yra matrica, kurios šaknys tenkina $(\sqrt{nP})^T\sqrt{nP} = nP$ ir $(\sqrt{nP})_i$ yra $\sqrt{nP}^1$ $i$ eilutė.

Sekamasis \textit{Kalman} filtras yra suvedamas į tokius punktus

\begin{enumerate}
    \item Turima $n$ eilės diskretinio laiko sistema
    \begin{equation}
        \begin{aligned}
            x_{k+1} &= f(x_k, u_k, t_k) + w_k \\
            y_k &= h(x_k, t_k) + v_k \\
            w_k &~ (0, Q_k) \\
            v_k &~ (0, R_k)
        \end{aligned}
    \end{equation}
    \item Filtras yra inicijuojamas
    \begin{equation}
        \begin{aligned}
            \hat{x}_0^+ &= E(x_0)\\
            P_0^+ &= E[ (x_0 - \hat{x}_0^+)(x_0 - \hat{x}_0^+)^T ]
        \end{aligned}
    \end{equation}
    \item Įvykdomos laiko atnaujinimo lygtis, kurios naudojamos apskaičiuoti būsenos įvertį ir kovariaciją iš vieno laiko matavimo į sekantį
    \begin{equation}
        \hat{x}_k^- = \frac{1}{2n} \sum_{i=1}^{2n} \hat{x}_k^{(i)}
    \end{equation}
    \begin{equation}
        P_k^- = \frac{1}{2n} \sum_{i=1}^{2n}( \hat{x}_k^{(i)} - \hat{x}_k^- )( \hat{x}_k^{(i)} - \hat{x}_k^- )^T + Q_{k-1}
    \end{equation}
    \item Atnaujinamas įverčio spėjimas
    \begin{equation}
        P_{xy} = \frac{1}{2n}\sum_{i=1}^{2n} (\hat{x}_k^{(i)} - \hat{x}_k^- )( \hat{y}_k^{(i)} - \hat{y}_k^- )^T
    \end{equation}
    \begin{equation}
        \begin{aligned}
            K_k &= P_{xy}P_y^{-1} \\
            \hat{x}_k^+ &= \hat{x}_k^- + K_k(y_k - \hat{y}_k) \\
            P_k^+ &= P_k^- - K_kP_yK_k^T
        \end{aligned}
    \end{equation}
\end{enumerate}

Algoritmas laikosi nuostato, kad proceso yra įverčio lygtis yra tiesinės ir įvertina triukšmą.
Bendru atveju proceso ir įverčio lygtis gali turėti ne tiesinio triukšmo, kuris ateina į procesą.

Sekamasis filtras gali gerokai pagerinti būsenos įverčio spėjimo našumą, lyginant su išplėstuoju \textit{Kalman} filtru ne tiesinėms sistemoms.
Papildomai galima paminėti, kad išplėstasis filtras reikalauja \textit{Jacobian} sprendimo, o šis filtras to nereikalauja.
Tai yra privalumas, kadangi kai kurioms sistemos \textit{Jacobian} skaičiavimas yra greitas ir paprastas, tačiau kitoms sistemoms tai yra labai sudėtingas uždavinys.


  \section{Literatūros apžvalga}

  Apžvalga


  \section{Praktiniai matavimai}

  \input{section/praktiniai_matavimai.tex}

  \section{Išvados}

  Žinių išgavimas yra procesas, kurio tikslas yra išgauti įdomius duomenų modelius iš didelio kiekio duomenų.
Tai taip pat yra vadinama žinių atradimo procesu ir jis tipiškai susideda iš duomenų valymo, integravimo, išrinkimo, transformavimo, modelio sukūrimo, jo įvertinimo ir gautų žinių atvaizdavimo.

Duomenų modelis yra įdomus jeigu jis galioja panaudojus testinius duomenis su tam tikru pasikliovimo lygiu, neįprastas ir potencialiai yra pritaikomas, paprastas suprasti žmogui.
Įdomūs duomenų modeliai sudaro žinias. 

Žinių inžinerija yra daugelio dimensijų požiūris į duomenis. Pagrindinės dimensijos yra duomenys, žinios, technologijos ir programos.

Žinių išgavimas gali būti suburtas iš skirtingo tipo duomenų, tol ko duomenis turi prasmę norimai sričiai, kaip duomenų bazės duomenis, duomenų saugyklos duomenis, perdavimo duomenis, ir pažengę duomenų tipai. 
Pažengę duomenų tipai susideda iš laikinės ar sekos duomenis, duomenų srautas, tūrinė informacija, tekstas ir daugialypės terpės duomenis, grafų ir tinklo duomenis.

Duomenų saugykla yra vieta saugoti duomenis ilgam laikotarpiui iš skirtingų šaltinų, bendroje schemoje.
Saugyklos sistemos pateikia daugelio dimensijų duomenis su analizės palaikymu, taip pat vadinamai realaus laiko analizės apdorojimui.

Daugelio dimensijų žinių išgavimas, integruoja pagrindinius žinių išgavimo metodus su realaus laiko daugelio dimensijų analize.
Metodas ieško įdomių modelių per skirtingas ir įvairias duomenų kombinacijas per skirtingas dimensijas, taikant skirtingas abstrakcijas, taip labai giliai analizuojant sąryšius tarp skirtingų dimensijų.

Žinių išgavimo funkcionalumai yra naudojami aprašant modelio arba žinių tipus, kurie buvo rasti per žinių išgavimą.
Tokie funkcionalumai yra charakterizavimas ir diskriminavimas, dažninių modelių išgavimas, asociacija ir koreliacija, klasifikavimas ir regresija, grupinė analizė; išskirčių aptikimas.

Žinių išgavimas turi labai daug pasisekusių įgyvendinimų, verslo srityje, interneto paieškoje, biologijos, sveikatos informatikoje, finansų, bibliotekos ir vyriausybės srityje.

  \newpage

  \bibliographystyle{plain}
  \bibliography{references}

\end{document}
