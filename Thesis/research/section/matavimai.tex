Vystant filtravimo modelį, labai svarbu yra sudaryti jutiklių matavimo planą.
Reikia nuspręsti kokiu scenarijumi remiantis bus surenkami duomenis iš jutiklių.
Naudojamas jutiklis turi tris pagreičio dimensijas.
Kiekviena iš dimensijų turi savo triukšmo komponentes.
Norint izoliuoti kiekvieną iš dimensijų atskirai, bei įvertinti triukšmo įtaką, matavimus pagal kiekvieną dimensiją reikia atlikti atskirai.
Tai reiškia, kad norint atlikti matavimus $x$ ašiai, reikia nedaryti įtakos kitoms ašims.
Tuomet matavimo planas susideda iš:

\begin{itemize}
    \item Veiki jutiklį pagal $x$ ašį
    \item Veiki jutiklį pagal $y$ ašį
    \item Veiki jutiklį pagal $z$ ašį
\end{itemize}

Matavimo planą dar labiau galim suprastinti, nedarant jokių matavimų pagal $z$ plokštumą ir nagrinėti sistemą tik dviejų dimensijų plokštumoje. 

\begin{figure}[H]
    \centering
    \begin{tikzpicture}
        \draw (0,0) node [left] {$x_1$};
        \draw[thick, black, *->] (0,0) -- (2,0) node [right] {$x_2$};

        \draw (2,-0.5) node [right] {$x_2$};
        \draw[thick, black, *->] (2,-0.5) -- (0,-0.5) node [left] {$x_3$}; 
        
    \end{tikzpicture}
    \caption{Jutiklio veikimas pagal $x$ ašį.}
    \label{tikz:x_axis_acceleration_measurement}
\end{figure}

Grafike \ref{tikz:x_axis_acceleration_measurement} yra parodytas matavimas pagal $x$ ašį.
Matavimas yra pradedamas nuo pozicijos $x_1$, po kažkurio laiko $\delta t_1$ jutiklis yra pajudinamas iki vietos $x_2$, stengiantis atlikti pakeitimą kaip galima tolygų.
Praėjus $\delta t_2$, jutiklis perkeliamas pagal $x$ ašį iš pozicijos $x_2$ į poziciją $x_3$.
Tokiu būdu gaunami duomenys, kuriuos galima panaudoti pozicijos nustatymui per matavimo laiką.

\begin{figure}[H]
    \centering
    \begin{tikzpicture}
        \begin{axis}[
            width=\textwidth,
            height=200,
            xlabel=$n$,
            ylabel=$g$
        ]
            \addplot[smooth, thick, dotted] plot table[x=time, y=ax] {data/x_1.data}; 
            \addlegendentry{$x$};
            \addplot[smooth, thick, dashed] plot table[x=time, y=ay] {data/x_1.data};
            \addlegendentry{$y$};
            \addplot[smooth, thick] plot table[x=time, y=az] {data/x_1.data};
            \addlegendentry{$z$};
        \end{axis}
    \end{tikzpicture}
    \caption{Jutiklio duomenys, kuomet yra atliekamas duomenų surinkimas $y$ ašiai.}
    \label{tikz:y_axis_acceleration_measurement}
\end{figure}

Grafike \ref{tikz:y_axis_acceleration_measurement} yra parodytas jutiklio duomenys, kuomet yra atliekamas matavimas pagal $y$ ašį.
Iš grafiko labai aiškiai matosi nuolatinė $y$ ašies komponentė, iš to galima spręsti, kad buvo nekorektiškai atliktas jutiklio kalibravimas.
Jutiklio kalibravimas yra atliekamas korektiškai, kuomet visų ašių duomenys pradedami nuo $\sim 0$ vertės.
Ką galima pastebėti panagrinėjus duomenis iš $x$ ir $z$ ašių.



