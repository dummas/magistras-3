Norint atlikti filtravimo operaciją, reikalingas tam tikras modelis, kuriuo pagalba bus atliekama pati operacija.
Modelis apibūdina nagrinėjamą sistemą, kokios komponentės, kaip priklauso nuo esamų parametrų.
Visas žinias apie sistemą turi pateikti modelį kurianti pusė.

\subsection{Pradinis modelis}

Naudojamas jutiklis yra pagreičio jutiklis.
Pagreičio jutiklis grąžina momentinį pagreičio pokytį duotuoju matavimo gavimo metu, $a(t)$.
Tokių duomenų realus pavyzdys pateikiamas \ref{fig:ax_accel} pav.

Turėdami pagreičio duomenis, laiko momentu galima gauti greičio duomenis $v(t)$, \ref{fig:ax_velocity} pav.

\begin{equation}
    v(t) = \int_0^t a(t) \delta t
\end{equation}

Pozicijos pokytis $s(t)$ priklauso nuo objekto greičio $v(t)$, \ref{fig:ax_distance} pav.

\begin{equation}
    s(t) = \int_0^tv(t) \delta t
\end{equation}

Iš to seka, kad norint gauti sistemos poziciją, pagreitį reikia integruoti du kartus, \ref{fig:ax_distance} pav.

\begin{equation}
    s(t) = \int_0^tv(t) \delta t = \int_0^t \int_0^t a(t) \delta t \delta t
\end{equation}

\begin{figure}[b]
    \centering
    \caption{Vienos ašies pagreičio jutiklio matavimo duomenys}
    \label{fig:ax_accel}
    \begin{tikzpicture}
        \begin{axis}[
            width=400pt,
            height=150pt,
            xlabel={$t, s$},
            ylabel={$m/s^2$},
            grid=major
        ]
            \addplot[smooth] table [x index = {0}, y index = {1}, col sep=comma] {data/acceleration.csv};
        \end{axis}
    \end{tikzpicture}
\end{figure}

\begin{figure}
    \centering
    \caption{Greičio duomenys, gauti integruojant pagreičio jutiklio vienos ašies duomenis}
    \label{fig:ax_velocity}
    \begin{tikzpicture}
        \begin{axis}[
            width=400pt,
            height=150pt,
            xlabel={$t, s$},
            ylabel={$m/s$},
            grid=major
        ]
            \addplot[smooth] table [x index = {0}, y index = {2}, col sep=comma] {data/acceleration.csv};
        \end{axis}
    \end{tikzpicture}
\end{figure}

\begin{figure}
    \centering
    \caption{Atstumo duomenys, gauti integruojant greičio duomenis pagal \ref{fig:ax_velocity}}
    \label{fig:ax_distance}
    \begin{tikzpicture}
        \begin{axis}[
            width=400pt,
            height=150pt,
            xlabel={$t, s$},
            ylabel={$m$},
            grid=major
        ]
            \addplot[smooth] table [x index = {0}, y index = {3}, col sep=comma] {data/acceleration.csv};
        \end{axis}
    \end{tikzpicture}
\end{figure}

Tokiu būdu gaunamas paprasčiausias matematinis modelis, kurio pagalba galima nustatyti dabartinę objekto poziciją.
Didžiausias šito modelio trūkumas, kuris matosi ir pateiktuose \ref{fig:ax_accel}, \ref{fig:ax_velocity} ir \ref{fig:ax_distance} pavyzdžiuose, yra triukšmo dauginimas dėl dvigubo integravimo.
Pagreitis turi labai didelį matavimo triukšmą.
Integruojant vieną kartą, triukšmo įtaka matavimui yra padidinama du kartus.
Integruojant du kartus -- triukšmas padidėja keturis kartus.

\subsection{\textit{Kalman} filtro modelis}

\textit{Kalman} filtras reikalauja modelį aprašyti priklausomybės lygčių sistema.
Sistema turi apibūdinti gaunamų duomenų priklausomybę nuo galutinės sistemos būsenos.
Aprašymas pradedamas nuo pagreičio matavimo matmens

\begin{equation}
    a(t) = a(t)
\end{equation}

Greitis priklauso nuo greičio, kuris buvo vienu laiko momentu prieš, ir nauju pagreičiu

\begin{equation}
    v(t) = v(t-1) + a(t) * \delta t
\end{equation}

Pozicija priklauso nuo prieš tai buvusios pozicijos, dabartinio greičio poslinkio, bei pagreičio kvadrato

\begin{equation}
    s(t) = s(t-1) + v(t-1) + \frac{a(t)}{2}\delta t^2
\end{equation}

Iš šitų lygčių yra sudaromas sistemos būsenos modelis

\begin{equation}
    \hat{x} = \begin{cases}
        s(t-1) + v(t-1) + \frac{a(t)}{2}\delta t^2 \\
        v(t-1) + a(t) * \delta t \\
        a(t)
    \end{cases}
\end{equation}

Kadangi iš viso mes turime tris ašis, modelis patrigubėja ir iš viso jį sudaro

\begin{equation}
    \hat{x} = [ x, y, z, \dot{x}, \dot{y}, \dot{z}, \ddot{x}, \ddot{y}, \ddot{z}]^T,
\end{equation}
kur $\dot{x}$ yra pozicijos pirmą išvestinė -- greitis, $\ddot{x}$ yra pozicijos antra išvestinė -- pagreitis.

Galutinė sistemos būsenos matrica atrodo

\begin{equation}
    F = \begin{bmatrix}
        1 & 0 & 0 & dT & 0 & 0 & 0.5*dT^2 & 0 & 0 \\
        0 & 1 & 0 & 0 & dT & 0 & 0 & 0.5*dT^2 & 0 \\
        0 & 0 & 1 & 0 & 0 & dT & 0 & 0 & 0.5*dT^2 \\
        0 & 0 & 0 & 1 & 0 & 0 & -dT & 0 & 0 \\
        0 & 0 & 0 & 0 & 1 & 0 & 0 & -dT & 0 \\
        0 & 0 & 0 & 0 & 0 & 1 & 0 & 0 & -dT \\
        0 & 0 & 0 & 0 & 0 & 0 & 1 & 0 & 0 \\
        0 & 0 & 0 & 0 & 0 & 0 & 0 & 1 & 0 \\
        0 & 0 & 0 & 0 & 0 & 0 & 0 & 0 & 1
    \end{bmatrix},
\end{equation}
kur $dT$ yra laikas, nusakantis kaip dažnai buvo daromi matavimai.

\textit{Kalman} filtro veikimui yra reikalauja turėti matavimų matricą.
Matavimų matrica nurodo, kuris iš modelio komponentų yra pats matavimas.
Šiuo atveju egzistuoja trys pagreičio komponentės, kurios nusako atliekamus matavimus.

\begin{equation}
    H = \begin{bmatrix}
        1 & 0 & 0 & 0 & 0 & 0 & 0 & 0 & 0 \\
        0 & 1 & 0 & 0 & 0 & 0 & 0 & 0 & 0 \\
        0 & 0 & 1 & 0 & 0 & 0 & 0 & 0 & 0
    \end{bmatrix}
\end{equation}

Pasirinkus matavimo matrica, toliau reikia pasirinkti dvi triukšmo įverčius.
Pirmas įvertis $q$, kuris nusako \textit{Kalman} filtro darbo proceso triukšmą.
Antras įvertis $r$ yra matavimo triukšmas. 
Matavimo triukšmo įvertį galima nustatyti dviem keliais. 
Pirmas kelias yra eksperimentinis -- atlikti stabilios matavimus ant stabilios platformos ir panaudoti gautą nuokrypį kaip matavimo triukšmą.
Kitas kelias yra pasikliauti gamintojo nurodytu matavimo skaičiumi.
Proceso triukšmas modelyje šiuo atveju yra pasirenkamas lygus matavimo triukšmui modelio paprastumui ir jie abu yra lygus $0.1$.

Turint modelio lygčių sistemą, matavimo matricą bei triukšmus, galima pradėti dirbi su \textit{Kalman} filtru.
Norint teisingai įvertinti \textit{Kalman} filtro darbą su duomenimis, visi įėjimo duomenys yra modeliuojami paprastu $sin$ signalu, pridedant atsitiktinį baltąjį triukšmą.
Signalo pavyzdys pateikiamas \ref{fig:sin_acc_data_with_noise} pav.
Kaip matosi iš grafiko, signalas pats turi labai daug triukšmo, tačiau seka labai specifinį kelią, kurį galima atsekti su paprastu vidurkinimu.
Toks signalas nėra pilnai atsitiktinis, jis turi kažkokį dėsnį.
Atlikus dvigubą originalaus signalo integravimą, gaunamas atstumo pokytis \ref{fig:sin_distance_data_with_noise} pav.
Tiesinį \textit{Kalman} filtro modelio rezultatas bus lyginamas su šiuo rezultatu ir bus laikomas teisingo rezultato etalonas.

\begin{figure}
    \centering
    \caption{Palyginimui naudojamas $sin$ signalas, su pridėtu baltuoju triukšmu}
    \label{fig:sin_acc_data_with_noise}
    \begin{tikzpicture}
        \begin{axis}[
            width=300pt,
            height=150pt,
            xlabel={$t, s$},
            ylabel={$m/s^2$},
            grid=major
        ]
            \addplot[smooth] table [x index = {0}, y index = {1}, col sep=comma] {data/linear_kalman.csv};
        \end{axis}
    \end{tikzpicture}
\end{figure}

\begin{figure}
    \centering
    \caption{Atstumo pokytis, dvigubai integruojant $sin$ signalą}
    \label{fig:sin_distance_data_with_noise}
    \begin{tikzpicture}
        \begin{axis}[
            width=300pt,
            height=150pt,
            xlabel={$t, s$},
            ylabel={$m$},
            grid=major
        ]
            \addplot[smooth] table [x index = {0}, y index = {3}, col sep=comma] {data/linear_kalman.csv};
        \end{axis}
    \end{tikzpicture}
\end{figure}

Pateikus į tiesinį \textit{Kalman} filtrą duomenis, kuris remiasi ankščiau aprašytomis lygtimis, gaunamas rezultatas, kuris yra pateikiamas \ref{fig:linear_kalman_filter_sin} pav.
Duotas grafikas pateikia esminį tiesinio \textit{Kalman} filtro naudojimo trūkumą duotos problemos atveju.
Esami duomenys visiškai neturi jokio tiesinės interpretacijos, todėl tiesinis \textit{Kalman} filtras šiuo atveju nėra tinkamas.
Reikia panaudoti tokį sprendimą, kurį galima būtų pritaikyti ne tiesinėms sistemoms.

\begin{figure}
    \centering
    \caption{Tiesinio Kalman filtro pozicijos pokyčio rezultatas, lyginant su pradiniais duomenimis}
    \label{fig:linear_kalman_filter_sin}
    \begin{tikzpicture}
        \begin{axis}[
            width=300pt,
            height=150pt,
            xlabel={$t, s$},
            ylabel={$m$},
            grid=major,
            legend pos=outer north east
        ]
            \addplot[smooth, dashed] table [x index = {0}, y index = {3}, col sep=comma] {data/linear_kalman.csv};
            \addplot[smooth] table [x index = {0}, y index = {4}, col sep=comma] {data/linear_kalman.csv};
            \legend{Pradinis pokytis, Tiesinio filtro pokytis}
        \end{axis}
    \end{tikzpicture}
\end{figure}

\subsection{Išplėstasis \textit{Kalman} filtras}

Išplėstasis \textit{Kalman} filtras yra netiesinis \textit{Kalman} filtras.
Tai yra pasiekiama, panaudojus \textit{Jacobian} matricą, kurios pagalba ne tiesinė lygtis yra ištiesinama viename taške.

Modelis, kuris yra pateikiamas išplėstajam \textit{Kalman} filtrui turi būti ne matricinės, o lygčių sistemos formos, kadangi bus atliekamas diferencialinis skaičiavimas.
Dėl šitos priežasties, nagrinėjamas modelis yra perrašomas į lygčių sistemą

\begin{equation}
    \hat{x} = \begin{cases}
        x_1 + x_2 + x_3 * 0.5 * dT^2 \\
        x_2 + x_3 * dT \\
        x_3
    \end{cases}
\end{equation}

Toliau reikia perrašyti matavimo gavimo matricą į lygtį.
Kadangi matavimas yra gaunamas tiesiogiai, tai ir pati lygtis yra tiesinė lygtis

\begin{equation}
    h(\hat{x}) = x_3;
\end{equation}

Sekantis žingsnis yra nustatyti triukšmus.
Norint palyginti išplėstąjį \textit{Kalman} filtro modelį su tiesiniu \textit{Kalman} filtro modeliu, pasirenkamos tokie patys triukšmų įverčiai.

Pateikus tokius pačius duomenis, kaip ir į tiesinį \textit{Kalman} filtrą, išplėstinio \textit{Kalman} filtro rezultatas yra pateikiamas \ref{fig:extended_kalman_filter_sin} pav.

\begin{figure}
    \centering
    \caption{Išplėsto Kalman filtro pozicijos pokyčio rezultatas, lyginant su pradiniais duomenimis}
    \label{fig:extended_kalman_filter_sin}
    \begin{tikzpicture}
        \begin{axis}[
            width=300pt,
            height=150pt,
            xlabel={$t, s$},
            ylabel={$m$},
            grid=major,
            legend pos=outer north east
        ]
            \addplot[smooth, dashed] table [x index = {0}, y index = {1}, col sep=comma] {data/extended_kalman.csv};
            \addplot[smooth] table [x index = {0}, y index = {2}, col sep=comma] {data/extended_kalman.csv};
            \legend{Pradinis pokytis, Išplėsto filtro pokytis}
        \end{axis}
    \end{tikzpicture}
\end{figure}

Lyginant tiesinio \textit{Kalman} filtro \ref{fig:linear_kalman_filter_sin} rezultato ir išplėstojo \textit{Kalman} filtro rezultato \ref{fig:extended_kalman_filter_sin} grafikus, galima teigti, jog išplėstasis \textit{Kalman} filtras veikia geriau.
Tai yra logiškas ir lauktas rezultatas, kadangi išplėstasis \textit{Kalman} filtras yra skirtas dirbti su ne tiesinėmis sistemomis \cite{6851407}.
Šiuo atveju, uždaviniui spręsti galima pasirinkti išplėstąjį \textit{Kalman} filtrą, tačiau tai yra ne paskutinis galimas problemos sprendimo būdas.
Šiuo metu labai pradeda populiarėti sekamasis \textit{Kalman} filtras, kuris sprendžia vieną pagrindinių išplėstojo \textit{Kalman} filtro problemų -- tiesinimo triukšmą.

\subsection{Sekamasis \textit{Kalman} filtras}

Sekamasis \textit{Kalman} filtro modelio reikalavimai yra lygiai tokie patys, kaip išplėstojo \textit{Kalman} filtro, todėl modelio perrašinėti nėra tikslo.
Triukšmo įverčiai yra paliekami tokie patys, kaip ir tiesinio bei išplėstinio filtro atvejais.
Modelio veikimo rezultatas yra pateikiamas \ref{fig:unscented_kalman_filter_sin} pav.

\begin{figure}
    \centering
    \caption{Sekamojo \textit{Kalman} filtro pozicijos pokyčio rezultatas, lyginant su pradiniais duomenimis}
    \label{fig:unscented_kalman_filter_sin}
    \begin{tikzpicture}
        \begin{axis}[
            width=300pt,
            height=150pt,
            xlabel={$t, s$},
            ylabel={$m$},
            grid=major,
            legend pos=outer north east
        ]
            \addplot[smooth, dashed] table [x index = {0}, y index = {1}, col sep=comma] {data/unscented_kalman.csv};
            \addplot[smooth] table [x index = {0}, y index = {2}, col sep=comma] {data/unscented_kalman.csv};
            \legend{Pradinis pokytis, Sekamo filtro pokytis}
        \end{axis}
    \end{tikzpicture}
\end{figure}

\subsection{Modelio pasirinkimas}

Sekamasis \textit{Kalman} filtras \ref{fig:unscented_kalman_filter_sin}, lyginant su išplėstuoju \textit{Kalman} filtru \ref{fig:extended_kalman_filter_sin} grafiškai neturi didelių privalumų.
Rezultato grafikas atrodo beveik identiškai.
Skirtumas matosi įverčiams artėjant prie nulinio pozicijos pokyčio, laiko momentu $t=6s$.
Išplėstasis Kalman filtras toje srityje turi pokytį tarp realios reikšmės ir filtruotos reikšmės, tuo tarpu sekamasis \textit{Kalman} filtras priartėja prie realios reikšmės, ko pasekoje mažina klaidos vertę.

Norint išmatuoti skaitinėmis reikšmėmis šių trijų pateiktų filtrų tikslumų skirtumus, bus panaudota kryžminė koreliacija.
Kryžminė koreliacija yra plačiai naudojamas matas, nustatyti panašumus tarp dviejų signalų.
Kuo gautas skaitinis įvertis yra arčiau nulio -- tuo signalai yra panašesni.
Verta paminėti, kad šitas matavimas yra naudojamas metodų palyginimui.

\begin{table}
    \centering
    \caption{Kryžminės koreliacijos matavimo rezultatas kiekvienam filtrui}
    \label{table:kalman_filter_comparison}
    \begin{tabular}{|c|c|c|} \hline
        Filtras & Koreliacija \\ \hline
        Tiesinis Kalman filtras & 0.8811 \\ \hline
        Išplėstas Kalman filtras & 0.9989 \\ \hline
        Sekamas Kalman filtras & 1.0000 \\ \hline
    \end{tabular}
\end{table}

Pagal gautus duomenis, kurie yra pateikiami \ref{table:kalman_filter_comparison} lentelėje galima daryti išvadas, kad geriausiai duomenis filtruoja būtent sekamasis \textit{Kalman} filtras ir netoli jo yra išplėstasis \textit{Kalman} filtras.
Remiantis šiuo rezultatu, filtro įgyvendinimui yra pasirenkamas sekamasis \textit{Kalman} filtras.

\subsection{Išvados}

Šiame skyriuje buvo apžvelgti trys filtro modeliai -- tiesinis, išplėstasis ir sekamasis \textit{Kalman} filtrai.
Palyginimas pradėtas nuo tiesioginio signalo dvigubo integravimo matavimo, kuris iš pirminės grafinės signalo analizės buvo nustatytas visiškai nepanaudojamas.
Pirmas realus nagrinėjamas paprastas \textit{Kalman} filtras, kuris reikalauja apibrėžti sistema lygčių sistema.
Tai yra žinios iš išorės, kuris yra reikalingos teisingai suderinti \textit{Kalman} filtro darbą.
Norint patikrinti \textit{Kalman} filtro veikimą, reikia į jį paduoti duomenis, kurių teisingas rezultatas yra iš anksto žinomas.
Dėl šios priežasties buvo pasirinktas paprastas sinuso tipo pagreičio signalas ir prie jo pridėtas baltasis triukšmas.
Tokiu būdu yra generuojamas signalas, kurio rezultatas yra iš anksto žinomas ir taip galima vertinti filtro darbą.
Atliktas matavimas parodė, kad tiesinis \textit{Kalman} filtras yra visiškai netinkamas darbo uždaviniui spręsti.
Sekantys du filtrai sprendžia paprasto \textit{Kalman} filtro pagrindinę problemą -- sistema turi būtinai būti tiesinė: išplėstas ir sekamasis \textit{Kalman} filtras.
Išplėstasis \textit{Kalman} filtras yra labai paplitęs filtras navigacinėse sistemose, todėl jis ir buvo pasirinktas kaip pirmas filtras, kuris sugeba dirbti su netiesine sistema.
Matavimas su žinomais duomenimis parodė, kad jis dirba geriau už paprastą \textit{Kalman} filtrą.
Sekamasis \textit{Kalman} filtras yra naujas filtro tipas, kuris po truputi įgauna populiarumą.
Jo matavimas parodė, kad jis veikia geriau už išplėstąjį \textit{Kalman} filtrą.
Dėl šios priežasties pasirenkamas sekamasis \textit{Kalman} filtras.


