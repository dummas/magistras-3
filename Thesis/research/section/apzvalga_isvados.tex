\subsection{Išvados}

Apžvalgos metu buvo išnagrinėti MEMS pagreičio ir giroskopo tipo jutikliai, aptarti kokie yra didžiausi jų privalumai ir trūkumai.
Tokio tipo jutikliai turi labai daug privalumų -- mažas dydis, svoris, patvari konstrukcija, mažos galios naudojimas.
Tačiau jie turi ir neigiamų savybių iš kurių svarbiausias yra triukšmas.
Triukšmo šaltiniai gali būti keletas -- nuolatinė dedamoji, jos stabilumas, atsitiktinis pokytis, temperatūros efektai bei kalibravimo klaidos.
Triukšmus galima mažinti naudojant įvairaus tipo filtrus.
Plačiausiai naudojami yra Kalman tipo filtrai.
Buvo išnagrinėti trys Kalman tipo filtrai, išnagrinėtas jų veikimo principai.
Pateiktos stiprios ir silpnos filtrų vietos.

Galiausiai buvo išnagrinėti du straipsniai, kurie sprendžia objekto pozicijos nustatymo problemą.
Pirmas straipsnis siūlo galimos navigacinės sistemos prototipą, remiantis interciniais jutikliais.
Prototipas buvo parengtas naudojant mobiliuosius įrenginius, kuriuose yra integruoti jutikliai.
Šitam darbe yra pažymima, jog naudoti vien tik pagreičio jutiklio duomenis pozicijos pokyčiui nusakyti yra be galo netikslu dėl dvigubos integracijos.
Dėl šios priežasties, jutiklis yra naudojamas kaip žingsnio indikatorius ir pozicijos pokytis yra skaičiuojamas pagal vidurinį žingsnio ilgį.
Darbe yra panaudotas išplėstas Kalman filtras, dėl to verta jį išbandyti darbo uždaviniui išspręsti.
Antras straipsnis taip pat naudoja inercinius jutiklius pozicijai nustatyti, o duomenų filtravimas vyksta taikant raiškios logikos algoritmą.
Joks kitas duomenų filtravimas nėra atliekamas.
Tai įrodo, kad nėra būtina naudoti didelio sudėtingumo algoritmus, norint sumažinti jutiklio triukšmus.

Šio darbo unikalumas yra sekamo Kalman filtro panaudojimas užduočiai spręsti. 
Išnagrinėti darbai dažnai taiko išplėstą Kalman filtrą, tačiau jis turi didelį minusą dėl vidurkinimo.
Šią problemą sprendžia sekamas Kalman filtras.
Straipsniai, kurie buvo nagrinėjami ir analizuojami, sekamo Kalman filtro panaudota niekur nebuvo.