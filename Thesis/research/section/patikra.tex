Norint patikrinti filtro įgyvendinimą įterptinėje posistemėje, reikia sudaryti kontroliuojamas sąlygas.
Sistemos įgyvendinimo tikslumas tikrinimas gali būti vykdomas tik viena ašimi, kadangi kiekviena ašis turi savo Kalman filtro modelį.
Tokiu būdu yra lengviau modifikuoti modelį, bei sutaupoma atminties.
Sprendimas yra padedamas ant stalo, lygiagrečiai padedant liniuotę $x$ ašiai ir pradedamas duomenų surinkimas.
Sistema yra slenkama tolygiai lygiagrečiai liniuotės $20~cm$ atstumu iki galo.
Palaukus porą sekundžių, įrenginys yra tolygiai traukiamas atgal į pradinę poziciją.
Gaunamas rezultatas yra pateikiamas \ref{fig:x_axis_measurement} pav.

\begin{figure}[b]
    \centering
    \caption{Atliktas \textit{x} ašies matavimas}
    \label{fig:x_axis_measurement}
    \begin{tikzpicture}
        \begin{axis}[
            width=300pt,
            height=150pt,
            xlabel={$t, s$},
            ylabel={$m$},
            grid=major,
            legend pos=outer north east
        ]
            \addplot[smooth, dashed] table [x index = {0}, y index = {1}, col sep=comma] {data/measurement.csv};
            \addplot[smooth] table [x index = {0}, y index = {2}, col sep=comma] {data/measurement.csv};
            \legend{Pradinis pokytis, Sekamo filtro pokytis}
        \end{axis}
    \end{tikzpicture}
\end{figure}

Analizuojant gautus matavimus, galima stebėti, kaip objektas pradžioje buvo vienoje vietoje, tuomet jo pozicija buvo pakeista į neigiamą ašies pusę, tuomet objekto pozicija nesikeičia.
Tai sąlygoja, kad objektas buvo nejudinamas.
Toliau objektas vėl juda atgal į pradinę poziciją.
Iš šito matavimo galima daryti išvadą, kad sekamas Kalman filtras darbą atlieka sėkmingai.
Grafike taip pat galima pastebėti, jog jutiklio normalizacija yra blogai atliekama ir atsiranda pastovioji komponentė.

Pastoviąja komponentę galima identifikuoti ilgiau surenkant jutiklio duomenys ir taip su daugiau duomenų per ilgesnį laiko tarpą identifikuoti konstantą.
Norint nustatyti ar konstanta buvo identifikuota sėkmingai, jutiklis yra paliekamas ramybės būsenoje ir renkami duomenys.
Atliktas matavimas yra pateikiamas \ref{fig:x_axis_measurement_long_normalization} pav.
Iš grafiko galima pamatyti, kad pradinis pokytis, kuris yra skaičiuojamas du kartus integruojant pagreičio pokytį, stabiliai keliasi į viršų.
Po nepilnos minutės jutiklis parodė, kad jo pozicija pasikeitė daugiau negu 8 metrus.
Kalman filtras sėkmingai identifikavo duotą komponentę ir neparodė jokio pokyčio.

\begin{figure}
    \centering
    \caption{Atliktas \textit{x} ašies matavimas, neveikiant objekto jokia jėga, atlikus ilgesnį normalizacijos procesą}
    \label{fig:x_axis_measurement_long_normalization}
    \begin{tikzpicture}
        \begin{axis}[
            width=300pt,
            height=150pt,
            xlabel={$t, s$},
            ylabel={$m$},
            grid=major,
            legend pos=outer north east
        ]
            \addplot[smooth, dashed] table [x index = {0}, y index = {1}, col sep=comma] {data/measurement_stovi.csv};
            \addplot[smooth] table [x index = {0}, y index = {2}, col sep=comma] {data/measurement_stovi.csv};
            \legend{Pradinis pokytis, Sekamo filtro pokytis}
        \end{axis}
    \end{tikzpicture}
\end{figure}

Sekantis matavimas yra pirmo matavimo kartotinis.
Objektas slenkamas viena ašimi į šoną, porą sekundžių palaukus objektas iš lėto gražinamas atgal į savo poziciją.
Atliktas matavimas yra pateikiamas \ref{fig:x_axis_measurement_sauna_long_normalization} pav.
Iš grafiko galima pamatyti, kad objektas sėkmingai rodo nulinį pozicijos pokytį pradžioje.
Po $4~s$ galima matyti, kad objekto pozicija pradėjo kisti.
Tuo metu objektas buvo veikiamas jėga ir stumiamas į vieną pusę.
Apie $5~s$ iš grafike galima matyti objekto sustojimą ir jokio pozicijos pokyčio neįžiūrėti.
Toliau apie $7~s$ objektas pradedamas grąžinti atgal.
Iki $8~s$ viskas atrodo teisingai, tačiau tuo pat metu atsiranda dar viena konstanta ir pozicijos pokytis tiesiškai didėja, nors objektas nėra veikiamas jokia išorine jėga.

\begin{figure}
    \centering
    \caption{Atliktas \textit{x} ašies matavimas, slenkant objektą, atlikus ilgesnį normalizacijos procesą}
    \label{fig:x_axis_measurement_sauna_long_normalization}
    \begin{tikzpicture}
        \begin{axis}[
            width=300pt,
            height=150pt,
            xlabel={$t, s$},
            ylabel={$m$},
            grid=major,
            legend pos=outer north east
        ]
            \addplot[smooth, dashed] table [x index = {0}, y index = {1}, col sep=comma] {data/measurement_sauna.csv};
            \addplot[smooth] table [x index = {0}, y index = {2}, col sep=comma] {data/measurement_sauna.csv};
            \legend{Pradinis pokytis, Sekamo filtro pokytis}
        \end{axis}
    \end{tikzpicture}
\end{figure}

Nauja konstanta gali atsirasti dėl kelių priežasčių.
Pirma priežastis yra giroskopo duomenų ignoravimas matavimo atlikimo metu.
Matavimas nėra atliekamas naudojantis tikslią įrangą, dėl to objektas gali pasisukti į kurią nors pusę.
Taip jutiklio ašis gali veikti žemės gravitacija ir tai įveda naują konstantą, kuri įtakoja gautus duomenis.
Antra priežastis yra pats jutiklis.
Tai yra labai pigus įrenginys ir atsitiktinis duomenų vaikščiojimas yra tikėtinas.

Kitais bandymais taip pat buvo pastebėta, kad normalizacijos procesas ne visuomet teisingai identifikuoja pastovios komponentės konstantą.
To priežastis yra normalizacijos proceso aplinka.
Įrenginys nėra pateikiamas ant labai stabilaus paviršiaus.
Darbo stalas perduoda labai daug vibracijų -- nuo kompiuterio ir monitoriaus ventiliatorių iki pelytės spaudimo.
To pasekoje, normalizacijos konstanta yra neteisinga ir pozicijos pokyčio grafikas rodo pastovų pozicijos pokytį į teigimą arba neigiamą pusę.

Įgyvendintas algoritmas įterptinėje sistemoje vieną įvertį apskaičiuoja per $0.017~s$ arba $17~ms$.
Programos dydis: kodas užima 13536B, tik skaitymo (RO) duomenys užima 700B, skaitymo ir rašymo duomenys (RW) užima 224B, o pradinių duomenų inicijavimas (ZI) 2488.
Turint šituos duomenis, galima apskaičiuoti 2712B sparčiosios atminties ir 14460B tik skaitymo atminties užimtumą.
