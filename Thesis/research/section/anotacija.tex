Maksim NORKIN

Objekto padėties nustatymas taikant MEMS jutiklius. 
Magistro baigiamasis darbas informatikos inžinerijos laipsniui. 
Vilniaus Gedimino technikos universitetas.
Vilnius \the\year, 69 p. 17 iliustracijų, 1 lentelė, 26 bibliotekų, 3 priedai.

Darbo tikslas yra sukurti objekto padėties nustatymo sistemą, taikant MEMS jutiklius ir mikrovaldiklį.
Pradžioje yra išnagrinėjami esami įgyvendinti funkcionalumai ir apibrėžiama, kokie šiuo metu yra taikomi sprendimai uždaviniui spręsti.
Atlikus analizę, sudarytas darbo planas išnagrinėti tris filtrus -- tiesinį (angl. \textit{linear}), išplėstinį (angl. \textit{extended}) ir sekamą (angl. \textit{unscented}) Kalman filtrus.
Išplėstas Kalman filtras industrijoje yra taikomas kaip standartas navigacinėse sistemose, sprendžiant netiesinio tipo uždavinius.
Visi trys filtrai buvo įgyvendinti ,,Matlab'' aplinkoje ir atliktas filtrų rezultatų palyginimas.
Buvo nustatyta -- blogiausiai netiesinį filtravimą atlieka tiesinis Kalman filtras, geriausiai - sekamas Kalman filtras.
Labai arti buvo išplėstas ir sekamas Kalman filtrai.
Po atliktos analizės sekamas Kalman filtras buvo įgyvendintas mikrovaldiklyje STM32.

