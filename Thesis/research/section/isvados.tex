Šiame darbe buvo aptarti MEMS tipo jutikliai ir galimas jų panaudojimas objekto pozicijos nustatymo tikslu. 
Pradžioje buvo atlikta jutiklių analizė, išnagrinėtas pagreičio ir giroskopo jutiklių darbas. 
Vertinant giroskopo jutiklio darbą nustatyta, kad matavimui atlikti yra naudojamas vibracinės kilmės elementas, kuris pasiduoda \textit{Coriolio} efektui. 
Vertinant šį reiškinį galima nustatyti kampinio pagreičio pokytį. 
Vertinant pagreičio jutiklio darbą nustatyta, kad yra du pagrindiniai įrenginio konstravimo būdai -- mechaniniai ir vibraciniai. 
Pagrindiniai MEMS tipo jutiklių privalumai, palyginti su kitos formos įtaisais, yra mažas dydis, svoris, pigi gamyba (esant dideliems kiekiams), patvari konstrukcija ir mažos galios naudojimas elektriniuose įtaisuose.

Kita medalio pusė yra didelis triukšmų šaltinių skaičius. 
Nuolatinės komponentės stabilumo, atsitiktinio vaikščiojimo klaidos, poveikis temperatūrai bei kalibravimo klaidos. 
Tiek pagreičio jutiklis, tiek giroskopas gali veikti vibracinio elemento pagrindu. 
Tokie elementai yra veikiami vibracijos triukšmo ir todėl matavimo metu atsiranda nuokrypių. 
Be to, bet koks fizinis kūnas reaguoja į temperatūros pokyčius. 
Tai irgi yra triukšmo šaltinis, nes skirtingos temperatūros sąlygomis vibracinis elementas vibruoja skirtingai. 
Tokios problemos kelia iššūkius tyrėjams kuriant tokias sistemas, kurios sugebėtų dirbti kuo efektyviau ir klaidų įtaka galutiniams skaičiavimams tikimybė būtų kuo mažesnė. 

Apžvelgti keli darbai, kurie siūlo spręsti nagrinėjamą uždavinį, tačiau naudojamos visiškai skirtingos metodikos tam tikslui išspręsti. 
Viename iš darbų naudojami tiek pagreičio, tiek kampinio pokyčio jutikliai. 
Kalibravimas atliekamas matuojant dreifą pirmas $6~s$. 
Toliau seka duomenys filtruojami, Kalman ir dalelių filtru. 
Abu filtrai galimai sprendžia navigacinę problemą. 
Kitame darbe remiamasi tik pagreičio jutikliu, naudojamas raiškios logikos elementas galutiniam pokyčio įverčiui rasti. 
Labai svarbu yra pažymėti, jog šitas darbas pasiūlė gerą jutiklių kalibracijos mechanizmą, remiantis gautais rezultatais (\ref{fig:fuzzy_logic_filter} pav.). 

Iš atliktos analizės galima spręsti, jog objekto pozicijos nustatymo sistemos yra reikalingos ir laukiamos rinkoje. 
Didžiausia kliūtis naudoti tokias sistemas yra įvairūs triukšmai. 
Tiesiogiai šitų duomenų naudoti modelyje tiesiog neįmanoma ir bet kokiu atveju reikia spręsti filtravimo problemą.

Darbe yra palyginami trys filtrai -- tiesinis, išplėstas ir sekamas Kalman filtrai.
Tiesinis Kalman filtras puikiai tinka spręsti tiesinio tipo uždavinius, tačiau netiesinėms sistemos jis visiškai netinka.
Norint panaudoti Kalman filtrą spręsti netiesiniams uždaviniams, modelio netiesinę funkciją reikia tiesinti.
Šitą problemą sprendžia išplėstinis Kalman filtras, kuris tiesina sistemą, naudojant Jacobian transformaciją ir sekamas Kalman filtras, kuris tiesina sistemą, naudojant sigma taškus.
Palyginus išplėstinio ir sekamo Kalman filtro darbo rezultatus, buvo pasirinktas sekamas Kalman filtras įgyvendinimui įterptinėje sistemoje.

Įgyvendinimui panaudota STM32 platforma su ,,Cortex-M4'' procesoriumi.
Darbui su matricomis buvo panaudota matematinė procesoriaus posistemė didžiausiai spartai užtikrinti.
Filtrui reikalingos papildomos funkcijos -- sigma taškų generavimas, \textit{Cholesky} dekompozicija buvo taip pat įgyvendintos.
Didžiausi įgyvendinimo sunkumai yra sunkus kodo skaitymas, darbui su matricomis trūkumas bei \textit{Cholesky} dekompozicijos įgyvendinimo nebūvimas.
Visas reikiamas funkcijas reikia įgyvendinti ir tikrinti jų veikimą.
Kita problema yra kodo skaitomumas.
Operacijos su matricomis yra atliekamos per pagalbines funkcijas.
Jokių pažįstamų matematinių ženklų nėra, todėl reikia labai atidžiai skaityti kas yra daroma su kintamuoju.
Buvo iškilusi problema dėl atminties valdymo.
Operacija, kuri kopijavo atminties sritį iš vieno adreso į kitą, kopijavo per daug.
Šios operacijos šalutinis poveikis įtakojo kitus kintamuosius, kurie visiškai atsitiktinai gaudavo reikšmes, kurių jie logiškai negali turėti.
Dėl šios priežasties buvo labai didelės laiko sąnaudos įgyvendinti sprendimą įterptinėje sistemoje.

Objekto padėties nustatymo įgyvendinimas buvo tikrinamas veikiant objektą jėga į vieną pusę ir po kurio laiko grąžinant jį į jo pradinę poziciją.
Matavimas buvo sudarytas kuo paprastesnis, taip nesunkiai nustatyti ar filtras veikia tinkamai.
Atlikti bandymai parodė normalizacijos proceso netikslumus.
Kiekviename matavime galima stebėti pastovios konstantos buvimą.
Prailginus normalizacijos procesą, ko pasekoje buvo surenkama daugiau įverčių konstantai identifikuoti, duomenys parodė, kad sekamas Kalman filtras veikia.
Tačiau po kurio laiko atsiranda dar viena konstanta ir dėl šios priežasties pozicijos pokytis pradeda didėti tiesine funkcija.
Nauja konstanta gali atsirasti dėl poros priežasčių, tačiau pirmiausiai reikia atlikti naudojamo modelio pakeitimus ir įtraukti kampinio pagreičio jutiklio duomenis.
Tokiu būdų, kai objektas pakeičia savo orientaciją plokštumoje, galima būtų kompensuoti naujas konstantas, kurias sukelia žemės gravitaciją.

Įgyvendintas algoritmas įterptinėje sistemoje vieną įvertį apskaičiuoja per $0.017~s$ arba $17~ms$.
Programos dydis: kodas užima $13536~B$, tik skaitymo (RO) duomenys užima $700~B$, skaitymo ir rašymo duomenys (RW) užima $224~B$, o pradinių duomenų inicijavimas (ZI) $2488~B$.
Turint šituos duomenis, galima apskaičiuoti $2712~B$ sparčiosios atminties ir $14460~B$ tik skaitymo atminties užimtumą.


