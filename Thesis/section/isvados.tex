Šiame darbe buvo aptartai MEMS tipo jutikliai ir galimas jų panaudojimas objekto pozicijos nustatymo procesui. Pradžioje buvo atlikta jutiklių analizė, panagrinėtas pagreičio ir giroskopo jutiklių darbas. Giroskopo atveju, matavimui yra naudojamas vibracinės kilmės elementas, kuris pasiduoda \textit{Coriolis} efektui. Iš šio reiškinio galima nuspręsti kampinio pagreičio pokytį. Pagreičio jutiklio atveju, yra du pagrindiniai įrenginio konstravimo būdai -- mechaniniai ir vibraciniai. Pagrindiniai MEMS tipo jutiklių privalumai prieš kitos formos įtaisus yra mažas dydis, svoris, pigi gamyba (esant dideliems kiekiams), patvari konstrukcija ir mažos galios naudojimas elektriniuose įtaisuose.

Kita medalio pusė yra didelis triukšmų šaltinių skaičius. Nuolatinės komponentės stabilumo, atsitiktinio vaikščiojimo klaidos, poveikis temperatūrai bei kalibravimo klaidos. Tiek pagreičio jutiklis, tiek giroskopas gali veikti gali veikti vibracinio elemento pagrindu. Tokie elementai yra veikiami vibracijos triukšmo ir įvelia į matavimą nukrypimus. Taip pat, bet koks fizinis kūnas reaguoja į temperatūrinius pokyčius. Čia irgi yra triukšmo šaltinis, kadangi prie skirtingų temperatūrų vibracinis elementas vibruoja skirtingai. Tokios problemos kelia iššūkius tyrėjams sugalvoti tokias sistemas, kurios sugeba dirbti kuo efektyviau ir mažinti klaidos įtaka galutiniams skaičiavimams.

Apžvelgti keli darbai, kurių tikslas yra toks pats, tačiau naudojamos visiškai skirtingos metodikos tam tikslui išspręsti. Pirmas nagrinėjimui darbas naudoja tiek pagreičio, tiek kampinio pokyčio jutiklius. Kalibravimas atliekamas matuojant dreifą pirmas $6~s$. Toliau seka duomenų filtravimas, Kalman ir dalelių filtru. Abu filtrai galimai sprendžia navigacinę problemą. Antras darbas remiasi tik pagreičio jutikliu ir naudoja raiškios logikos elementą galutiniam pokyčio įverčiui rasti. Labai svarbu yra pažymėti, jog darbas pasiūlė gerą jutiklių kalibracijos mechanizmą, remiantis gautais rezultatais (\ref{fig:fuzzy_logic_filter} pav.). 

Iš atliktos analizės galima spręsti, jog objekto pozicijos nustatymo sistemos yra reikalingos ir laukiamos rinkoje. Didžiausia kliūtis naudoti tokias sistemas yra įvairūs triukšmai. Tiesiogiai šitų duomenų naudoti modelyje tiesiog nėra galima ir bet kokiu atveju reikia spręsti filtravimo uždavinį.

