The Blue Gene/Q sistema yra trečia Blue Gene šeimos kompiuterių architektūros karta \cite{gilge2014ibm}.
Sistema sujungia daugybe komponentų, tarp kurių yra vienas ar daugiau skaičiavimo spintų ir neprivalomi Į/I spintos.
Sistema tankiai susideda iš skaičiavimo mazgų, Į/I spintos ir aptarnavimo kortelės.
Papildoma geležis yra susiejama su saugojimo posisteme, pagrindinis aptarnavimo mazgu (SN), išorinio priėjimo mazgu (FENs) ir komunikacijos posisteme.
Į/I stalčiai, kurie susideda Į/I mazgų, sujungtu su funkciniu vietiniu tinklu (LAN) bendravimui su bylų serveriais, FENs ir SN.
Aptarnavimo kortelės sujungtos su valgymo potinkliu ir yra naudojami SN Blue Gene/Q geležies kontrolei.

Aptarnavimo mazgas pateikia vieną tašką, per kurį galima kontroliuoti ir administruoti Blue Gene/Q sistemą.
Yra įmanoma operuoti sistema, kuri susideda tik iš vieno aptarnavimo mazgo.
Tačiau sistemos aplinka taip pat gali būti sukonfigūruota įtraukti potinklio aptarnavimo mazgus, didesniam plečiamumui.

Priekinio priėjimo mazgui, kuris taip pat vadinamas kaip prisijungimo mazgas, pateikia sistemos resursus, kurie programinės įrangos kūrėjui leidžia prisijungti prie sistemos.
Programuotojai redaguoja ir kompiliuoja programinius paketus, sukuria darbus ir kontrolės bylas, paleidžia darbus sistemoje, nagrinėja gautos programos rezultatus ir atlieka kitus darbus.

Skaičiavimo kortos susideda iš 16 IBM Blue Gene/Q PowerPC A2 branduolio procesorių ir 16 GB darbinės atminties.
Trisdešimt dvi tokios kortos gali būti įtrauktos į mazgo plokštę ir 16 mazgo kortos yra laikomos midplane.
Kompiuterio dėžė gali būti sukonfigūruota dirbti tiek su vienu, tiek su puse midplane.
Sistema gali būti išplėsta iki 512 kompiuterių dėžės.

Kompiuterių dėžių komponentai yra vėsinami arba vandeniu arba oru.
Vanduo yra naudojamas apdorojimo mazguose.
Oras yra naudojamas maitinimo tiekime ir Į/I stalčiuose, kurie yra prijungti prie sistemos.

Į/I stalčiai yra arba atskiros dėžės arba statomos ant kompiuterių dėžių, dažnai vadinamos viršaus kepurės.
Aštuoni Į/I mazgai yra talpinami kiekvienoje iš Į/I stalčių.
Kompiuterinės dėžės yra konfigūruojamos iki keturių Į/I stalčių, du per midplane, naudojant viršutinę kepurę.
Į/I stalčių pozicija Į/I dėžėse yra reguliuojama didelių sistemų diegime, kur Į/I mazgų skaičius negali būti talpinamas spintoje.




