Naudojama operacinė sistema yra ``Compute Node Kernel (CNK)'', kuri yra panaši į Linux ir suteikia naudotojui darbo aplinką leisti procesus \cite{milano2013ibm}.
CNK įtraukia tokias paslaugas:

\begin{itemize}
    \item Proceso sukūrimas ir valdymas
    \item Atminties valdymas
    \item Proceso skrodimas
    \item Patikimumas, pasiekiamumas ir universalumo valdymas
    \item Bylų Į/I
    \item Tinklas
\end{itemize}

Programinis paketas įtraukia į save standartines C/C++ ir Fortran darbo bibliotekas.
Kiek įmanoma, yra palaikomos ir atviro standarto Portable Operacing System Interface (POSIX) sąsajas.
CNK turi grubų gijų palaikymą, kuris įtraukia tokias technologijas kaip pthread, XL OpenMP ir GNU OpenMP.
Native POSIX Thread Library pthreads įgyvendinimas yra palaikomas be modifikacijų.
Statiškai sujungtos paleidžiamos programos pateikia optimalų veikimą, tačiau CNK taip pat palaiko ir dinamiškai sujungtas paleidžiamas programas.
Iš dinamiškai sujungtų programinių programų yra palaikoma tik Python.

\subsection{Į/I mazgų branduolys ir paslaugos}

Į/I mazgas dirba ant branduolio, kuris yra pakeistas Red Hat Enterprise Linux 6 branduolys.
Modifikacija pateikia platformos palaikymą ir didesnį našumą.

Į/I mazgo programinis paketas suteikia Į/I paslaugas skaičiavimo mazgams.
Pavyzdžiui, programos, kurios dirba ant skaičiavimo mazgų gali prieiti prie bylų serverių ir komunikuoti su procesais kitose mašinose.
Į/I mazgai taip pat gali turėti svarbų vaidmenį paleidžiant ir sustabdant darbus ir koordinuojant jų darbą su skrodimo ir stebėjimo įrankiais.

Blue Gene/Q yra be disko sistema, todėl bylų serveriai turi būti.
Aukšto našumo lygiagreti bylų sistema turi būti.
Blue Gene/Q yra labai lanksti ir palaiko skirtingas bylų sistemas, kurias palaiko Linux.
Tipinė lygiagreti bylų sistema yra IBM General Parallel File System (GPFS) ir Lustre.

Į/I magzai įtraukia pilną IP, TCP ir UDP protokolo įgyvendinimus.
Šių protokolų poaibės yra pasiekiamos naudotojo procesams, kurie veikia ant apdorojimo mazgų ir turi susiejimą su Į/I mazgu.
Programos procesai bendrauja su procesais, kurie veikia kituose sistemose per kliento pusės jungtis.
Serverio pusės jungtys taip pat yra palaikomos.
Į/I mazgo jungčių įgyvendinimas yra atliktas tokiu būdu, kad procesai iš apdorojimo mazgų, mano esą tiesiai ant Į/I mazgo.
Tai reiškia, kad jungties port skaičius yra vienas adresas per grupę.
Skaičiavimo mazgai dalinasi Į/I mazgo IP adresu.

Į/I branduolio mazgas yra įgyvendintas taip, kad jis būtų kraunamas kuo rečiau.
Krovimo procesas įtraukia krovimą į darbinę atmintį Linux branduolį ir jo paleidimą iš tos pačios atminties.
Darbinės atminties įkrautas branduolys tuomet yra pateikiamas pradinei bylų sistemai.
Sistema sudedama iš pagrindinių komandų, kurios yra skirtos pakrauti bylų sistemą ant aptarnavimo mazgo per Network File System (NFS).
Krovimas toliau tęsiasi paleisdamas paleidžiant paleidimo programas iš NFS.
Jis taip pat paleidžia naudojo parengtus paleidimo programas, atliekant specifinius veiksmus, kaip pavyzdžiui sistemos pranešimų konfigūracija ir jungimas aukšto našumo bylų sistemas.

Bendros įrankių dėžės bibliotekos ir visi pagrindiniai Linux teksto ir kiauto įrankiai egzistuoja darbinėje atmintyje.
Paketai, kaip pavyzdžiui GPFS, ir naudotojo pateiktos programos yra prijungiamos per NFS ir valdomos su administracinėmis teisėmis.

\subsection{Žinučių perdavimo sąsaja}

Žinučių perdavimo sąsajos įgyvendinimas Blue Gene/Q sistemoje yra MPICH2 standartas, kuris buvo sukurtas Argonne National Labs.

Dinaminis proceso valdymo funkcija yra MPI-2 standartas ir nėra palaikoma Blue Gene/Q sistemos.
Tačiau skirtingi gijų modeliai yra palaikomi.

\subsection{Kompiliatoriai}

Sistemos įrankių dėžės kompiliatoriai ir IBM XL kompiliatoriai Blue Gene/Q skaičiavimo sistemos mazgams yra galimi naudoti iš sistemos.
Kadangi kompiliavimas yra atliekamas priekinio priėjimo mazguose ir ne ant Blue Gene/Q sistemos, yra naudojami daugybės platformos kompiliatoriai.

\subsubsection{GNU kompiliatoriai}

Kompiliatoriai yra paremti GNU šeimos kompiliatoriais.
Kuomet yra diegiamas programinis paketas į sistemą, RPM paketų valdyklė yra suteikiama, kad naudotojai galėtų sukurti su sudiegti įrankių dėžę į \textit{gnu-linux} direktoriją.
Tie patys kompiliatoriai yra panaudoti ir kuriant pačios Blue Gene/Q sistemos programas, todėl yra galimas pilnas bibliotekų palaikymas naudotojo programoms.

\subsubsection{IBM XL kompiliatoriai}

IBM XL kompiliatoriai Blue Gene/Q sistemai gali būti panaudoti programų kompiliavimui.
Kompiliatorius gali sutekti aukštesnio lygio optimizavimo lygį, lyginant su GNU kompiliatoriais.
XL kompiliatoriai palaiko vienos instrukcijos, daugybės duomenų vektorizavimą.
Vektorizavimas suteikia automatinį kodo generavimą ketvirtos eilės slankaus skaičiaus naudojimui.
Toks vienetas vienu metu gali atlaikyti keturias slankaus skaičiaus instrukcijas.
XL kompiliatorius taip pat palaiko išeities kodo sintaksę, kurios pagalba galima skaidriau naudoti atmintį ir spekuliacines gijas.

\subsubsection{MPI apgaubimo programos}

MPI apgaubimo programos yra kompiliatoriaus apgaubimo programos, kurios suteikiamos Blue Gene/Q tvarkyklėje.
Tokios programos gali būti panaudotos kompiliavimui ir sujungimui programų, kurios naudoja MPI.
Skirtinos MPI programos yra galimos, priklausomai nuo panaudoto kompiliatoriaus ir nuo panaudotų bibliotekų versijos.
Apgaubimo programos paleidžia reikalingą kompiliatorių ir sudeda visas reikalingas direktorijas, bibliotekas, ir parametrus, kurie yra reikalingi MPI programos kompiliavimui.
Kiekvienai kompiliavimo kalbai ir Blue Gene/Q standartui, yra atitinkama MPI programa.
Taip pat egzistuoja saugios gijos XL kompiliatoriaus versijos.

\subsection{Programos vystymas ir skrodimas}

Programos vystymo specialistai gali prieiti prie išorinės pusės mazgų, kad sukompiliuoti ir skrosti programas, kurios buvo priskirtos į sistemos darbo režimą ir atlikti kitus interaktyvius veiksmus.

\subsubsection{Skrodimo programos}

Sistema palaiko GNU Project Debugger (GDB) paleidimą su programomis, kurios veikia ant skaičiavimo mazgų.

\subsubsection{Programų paleidimas}

Sistemos programos gali dirbti skirtingais režimais.
Dažniausiai naudojamas metodas yra naudoti darbo planuotoją, kuris palaiko Blue Gene/Q sistemą, kaip pavyzdžiui LoadLeveler planuotojas.
Kita mažiau naudojama alternatyva yra naudoti \textit{runjob} komanda tiesiogiai.
Visi planuotojai naudoja \textit{runjob} sąsaja darbų tvirtinimui, tačiau planuotojai taip pat gali apgaubti darbą su kita komanda arba darbo pridavimo sąsaja.

\subsubsection{Programos atminties sąlygos}

Sistemoje, visa fizinę atmintis ant vieno skaičiavimo mazgo yra $16~GB$, todėl rašant programinį paketą reikia turėti šitą informaciją omenyje.
Kai kuri iš šios atminties jau yra užimta CNK.
Dalinama atminties vieta taip pat yra paskiriama naudotojo procesui, kuomet procesas yra sukuriamas.

CNK seka steko ir kaupo susidūrimus, kuomet kaupas yra plečiamas su \textit{brk()} ir \textit{mmap()} sistemos kvietimais.
CNK ir jos privatūs duomenys yra apsaugomi nuo naudotojo procesų ar gijų skaitymo ir rašymo operacijų.
Proceso kodo erdvė irgi yra apsaugoma nuo kitų procesų ar gijų.
Kodas ir skaitymo duomenys yra padalinami tarp procesų, kurie dalinasi tą patį mazgą.

Atminties dydis, kuris yra reikalingas programai yra svarbus dalykas Blue Gene/Q sistemai.
Naudojama programos atmintis yra klasifikuojama:

\begin{itemize}
    \item bss -- ne inicializuoti statiniai ar paprasti kintamieji
    \item data -- inicializuoti statiniai ar paprasti kintamieji
    \item heap -- kontroliuojamas priskyrimo masyvas
    \item stack -- kontroliuojamas automatinis masyvas ir kintamieji
    \item text -- programos tekstas (instrukcijos) ir skaitymo duomenys
\end{itemize}

Blue Gene/Q sistema palaiko 64 bitų atminties modelį.
Galima naudoti Linux \textit{size} komandą, parodyti programos naudojamą atminties dydį.
Tačiau, \textit{size} komanda neparodo jokios informacijos apie paleisto proceso naudojamos steko ir kaupimo  atminties dydį.

Galimas atminties dydis, kuris yra suteikiamas programai, priklauso nuo procesų skaičiaus viename mazge.
Galimos $16~GB$ atminties dydis yra paskirstomas kaip įmanoma lygiai kiekvienam procesui, kuris dirba ant to mazgo.
Kadangi atmintis yra ribotas resursas, dažniausiai rekomenduojama konservuoti programos atmintį.
Kai kuriais atvejais, atminties reikalavimas gali būti sumažintas paskirstant duomenis, kurie buvo replikuoti originaliam kode.
Tačiau tai reikalauja papildomos komunikacijos.
Sistemoje veikiančių procesų skaičius gali būti labai didelis.
Atmintį galima įsivaizduoti kaip dviejų dimensijų masyvą, kur viena dimensija yra procesų skaičius, o kita dimensija -- galimas atminties laukų skaičius.

\subsubsection{Kiti pastebėjimai}

Labai svarbu yra suprasti, jog operacinė sistema, kuri randasi ant skaičiavimo mazgo nėra pilnas Linux branduolys.
Todėl reikia būti labai atsargiems tokiose operacijose:

\begin{itemize}
    \item Įėjimas ir išėjimas. Labai svarbu kreipti į tai dėmesį. CNK neatlieka jokio Į/I -- viskas yra atliekama Į/I mazgo
    \item Bylų Į/I. Yra palaikoma labai limituota bylų sąsaja. Negalima naudoti asinchroninių bylų operacijų, kadangi tai sukelia klaidų darbo metu
    \item Standartinė įvestis yra palaikoma Blue Gene/Q
    \item Jungčių kvietimai yra palaikoma sistemos
    \item Tiek statinis, tiek dinaminis jungimas yra palaikomas sistemos.
    \item Kiauto programos nėra palaikomos CNK branguolio. Tik paleidžiama programa gali būti startuojama.
    Todėl, jeigu programa reikalauja kiauto programos srautui kontroliuoti, srautas turi būti adaptuojamas.
\end{itemize}
