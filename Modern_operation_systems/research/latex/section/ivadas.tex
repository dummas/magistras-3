Mira yra peta eilės ``Blue Gene/Q'' superkompiuteris \cite{wiki:IBM_Mira}.
2013 metų birželį, jis buvo patalpintas į top500 kaip penktas greičiausias superkompiuteris pasaulyje.
Jis sugeba dirbti $8.59$ peta flopų našumu ir suvartoja $3.9~MW$ elektros energijos.
Superkompiuteris buvo sukonstruotas IBM kompanijos ``Arhonne National Laboratory'' reikmėms, kuri vykdo Jungtinės Amerikos energetikos tyrimus, ir dalinai yra finansuojama ``National Science Foundation''.
Mira yra naudojama moksliniams tyrimams, tarp kurių yra medžiagų mokslas, klimatologija, seismologija ir kompiuterinė chemija.
Superkomoiuteris yra utilizuotas viduje šešiolikai projektų, kurie yra parinkti Energetikos departamento.

Superkompiuterio būstinė buvo įkurta ``America COMPETES Act'', pasirašyta prezidento Bušo 2007 metais ir prezidento Obama 2011 metais.
Jungtinių Amerikos Valstijų kirtis ant super-kompiuterių yra žvelgiamas kaip atsakas Kinijos progresui šioje srityje.
Kinijos \textit{Tianhe-1A}, kuris randasi ``Tianjin National Supercomputer Center'' yra vertinamas kaip pats galingiausias super-kompiuteris pasaulyje nuo 2010 spalio iki 2011 birželio.
Mira, kartu su ``IBM Sequoia'' ir ``Blue Waters'' yra tris amerikos peta dydžio superkompiuteriai, kurie buvo paleisti 2012 metais.

Superkompiuterio Mira statymo kaštai nebuvo atskleisti IBM.
Ankstyvus pranešimai spėja, kad tokio kompiuterio statybos galėjo atsieiti apie $\$50$ milijonų ir ``Argonne National Laboratory'' pranešą, kad visas kompiuterio pirkimo biudžetas sudarė iki  $\$180$ milijonų.