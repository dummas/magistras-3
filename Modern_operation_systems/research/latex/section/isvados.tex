Mira super kompiuteris yra didelio galingumo ``Blue Gene/Q'' superkompiuteris, kuris leido Jungtinėms Amerikos Valstijoms pateikti atsakymą į Kinijos progresui šioje srityje. Tikslios kompiuterio sudedamosios dalys nėra pateikiamos, tačiau yra žinoma, kad sistema privalo sudaryti skaičiavimo mazgai, kartu su aptarnavimo kontrole ir nebūtinai turi sudaryti Į/I mazgai. Aptarnavimo kontrolės sujungtos su valdymo potinkliu ir yra naudojami Blue Gene/Q aparatinės įrangos kontrolei.

Sistema neleidžia tiesioginį priėjimą prie sistemos. Tam yra skirti priekinio priėjimo mazgai, arba prisijungimo mazgai, kurie pateikia sistemos resursus ir įrangos kūrėjui leidžia prisijungti prie sistemos. Ten pat yra atliekamas programų inžinierių darbas, kurie jie gali redaguoti, kompiliuoti programinius paketus, bei pateikti darbus super-kompiuteriui.

Visas super-kompiuteris naudoja specialią operacinę sistemą, kuri vadinama CNK, yra paremta Linux operacine sistema ir suteikia naudotojui aplinką leisti procesus. Konkreti Linux distribucija yra Red Hat Enterprise Linux 6. Atlikti pakeitimai leidžia palaikyti platformą ir suteikti didesnį našumą.

Kaip jau buvo minėta, Į/I mazgas suteikia priėjimą skaičiavimo mazgams sąsaja su Į/I. Šitie mazgai suteikia skaičiavimo mazgams galimybę prieiti prie bylų serverių ir komunikuoti su procesais kitose mašinose.

Skaičiavimo mazgas palaiko labai daug sąsajų, kurios leidžia efektyviai atlikti skaičiavimo operacijas ir turėti sąryšius per žinučių sąsaja. Specialiai skaičiavimo mašinai yra pateikiami kompiliatoriai, kurie skirti efektyviau leisti programinius paketus ant duotos bazės. Be sąsajų palaikymo, CNK taip pat užsiima atminties valdymu, jos priėjimu vietiniam mazge, jungčių sujungimais, signalais, gijų valdymu.

Super kompiuteris yra naudojamas išskirtinai moksliniams tyrimams, kuriems reikalingi labai tikslios ir plačios simuliacijos.