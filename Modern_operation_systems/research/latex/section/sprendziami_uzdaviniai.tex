Argonne mokslininkai panaudojo Mira super kompiuterį, kad identifikuoti ir patobulinti naują mechanizmą, kuris leidžia pašalinti trintį. To pasekoje pirmą kartą buvo sukurta nauja hibridinė medžiaga, kuri paveldi didelį tepumą makro lygmenyje.

% https://www.sciencedaily.com/releases/2015/07/150721194001.htm
 
Modeliuojami defektai Ni-Al su EAM ir DFT skaičiavimais \cite{0965-0393-24-4-045012}. Atliktas detalus palyginimas tarp įterpto atomo modelio (EAM) ir paskirstymo funkcijos teoremos (DFT) skaičiuojant Ni lydinio defektų sistemas.

Sukimo defektų plačiajuosčiam dažnių diapazone puslaidininkiuose \cite{Seo2016}.

Kaip yra matoma iš pateikiamų straipsnių -- Mira yra naudojamas išskirtinai moksliniams tyrimams, kuriems reikalingos didelės simuliacijos duomenims pasitvirtinti ir prognozuoti.

% http://www.alcf.anl.gov/publications