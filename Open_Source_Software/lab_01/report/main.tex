\documentclass[11pt, a4paper, lithuanian]{article}

\usepackage[left=25mm,right=15mm,top=15mm,bottom=15mm]{geometry}
\usepackage[utf8x]{inputenc}
\usepackage[L7x]{fontenc}
\usepackage[lithuanian]{babel}
\usepackage{listings}
\usepackage{amsmath, amssymb}
\usepackage{graphicx}
 \usepackage{url}

\author{AKSfm-15, Maksim Norkin}
\title{Vaizdo atpažinimas}

\lstset{
  language=Matlab,
  basicstyle=\footnotesize,
  columns=fixed,
  numbers=none,
  showspaces=false,
  xleftmargin=20pt
}

\begin{document}

    \maketitle

    \section{Užduotis}

    Parinkti atviro kodo virtualią mašiną ir ją įdiegti. Virtualioje mašinoje įdiegti atviro kodo operacinę sistemą. Operacinėje sistemoje įdiegti git serverį.

    \section{Atlikimo žingsniai}

    \section{Virtualios mašinos parinkimas, licencija}

    Virtualios mašinos valdymo sprendimas pasirinktas VirtualBox programinis paketas, kurios licencija yra GPLv2. 
    Licencija leidžia naudoti programinį paketą kaip norima, svarbiausia yra pažymėti kas yra naudojamos programinės įrangos autorius.

    VirtualBox yra galingas x86 ir AMD/Intel64 virtualizacijos produktas, kuris yra tinkamas tiek naudoti namie, tiek darbe. 
    Sistema yra labai gausi funkcijomis, kurias galima naudoti darbo aplinkoje. 
    Tai yra vienintelis tokio tipo atviro kodo sprendimas, kuris leidžiamas naudoti profesionalioje aplinkoje.

    VirtualBox sistema veikia ant Windows, Linux, Macintosh ir Solaris operacinių sistemų.
    Naudojantis sprendimu, galima paleisti didelį spektrą operacinių sistemų, įtraukiant Windows, DOS/Windows 3.1, Linux, Solaris, OpenSolaris, OS/2 ir OpenBSD.
    VirtualBox yra aktyviai vystomas ir dažnai išleidžiami atnaujinimai.
    Jo funkcijų skaičius pastoviai yra didinamas. Taip pat yra atliekami darbai prie didesnio palaikomų operacinių sistemų palaikymo.
    Sprendimas vystomas bendruomenės, tačiau yra palaikomas ir poros įmonių. 
    Kiekvienas yra agituojamas prisidėti prie šito produkto vystymo, o Oracle pasirūpina, kad produktas atitiktų profesionalios aplinkos keliamus reikalavimus.

    \section{Atviro kodo operacinės sistemos parinkimas, licencija, virtualios mašinos konfigūracija}

    Pasirinkta Debian operacinė sistema, dėl distribucijos stabilumo. 
    Operacinė sistema yra Unix tipo kompiuterinė operacinė sistema, kuri visiškai susideda iš nemokamų programų, daugelis kurių yra licencijuoti pagal GNU GPL.
    Programinės įrangos sudėliojimu rūpinasi individų grupė, kuri formuoja Debian Project.
    Egzistuoja trys pagrindinės sistemos atšakos: stabili, testuojama ir nestabili.

    Debian stabili atšaka yra viena populiariausių operacinių sistemų, kuri yra naudojamas asmeniniuose kompiuteriuose ir serveriuose.
    Jinai yra naudojama kaip bazė daugelio kitų distribucijų.

    \section{GIT serverio įdiegimas atviro kodo operacinėje sistemoje (žigsniai)}


\end{document}
