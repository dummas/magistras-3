Užduoties tikslas yra sukurti programinį paketą \textit{Cortex-M4} pagrindu veikiančiai įrangai, kuri sugebėtų komunikuoti su \textit{MPU 9250} devynių ašių MEMS jutikliu ir atiduotu gautus duomenis į pagrindinį kompiuterį.
Pagrindinis kompiuteris elgiasi kaip duomenų priėmimo mazgas ir tolimesnis jų skirstytojas.
Kaip tolimesnis mazgas duomenų apdorojimui yra programa, kuri atlieka pirminį duomenų filtravimą ir pateikia gautus duomenis sistemos naudotojui tolimesnei analizei.

Kuriant programine įrangą yra tikslas pritaikyti atviro kodo licencija, kad būtų užkirstas kelias piktybiniam darbo išnaudojimui, bei apsisaugoti nuo bet kokių teisinių ir socialinių ginčų.
Įterptinių sistemų bendruomenėje programinės įrangos atžvilgiu vyrauja atviro kodo principas.
Dėl šios priežasties atliktam darbui norima pritaikyti kuo atviresnę licenciją, kad nebūtų jokių ribojimų kodo panaudojime.
Dėl tos pačios priežasties, labai paprastėja licencijos pritaikymo uždavinys.
Prieš pritaikant licenciją, reikia patikrinti ar naudojamos programinės įrangos licencija leidžia keisti prieš tai buvusią licenciją.
Kuomet yra pritaikyta labai atvira licencija, kuri visiškai neriboja licencijavimo pagrindus, tuomet pakeistai programinei įrangai licencija pritaikyti yra labai paprasta. 