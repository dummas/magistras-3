Analogiškos programinės įrangos analizė suteikia galimybę detaliau įsivertinti rinkos situaciją ir surasti sprendimą, kuris gali tapti baze kuriamai programinei įrangai.
Kaip alternatyva yra bibliotekų panaudojimas.
Jeigu rastos programinės įrangos licenciją yra nesuderinama su norima naudoti licencija -- galima peržiūrėti naudojamų bibliotekų licencijas, ir jeigu bibliotekų licencijos yra suderinamos su norima naudoti licencija -- panaudoti būtent tas bibliotekas, taip palengvinant visą darbą.

\subsection{Analogiška programinė įranga}

Paprasta programa, kuri yra labai gerai pritaikoma esamai platformai yra \cite{MPU9243:online}. Programinė įranga yra licencijuojama pagal \textit{Beerware} licenciją. Šita licencija leidžia naudoti išeities kodą arba pačią programinę įrangą laisvai, bet kokias tikslais.

\begin{verbatim}
 * MPU9250 Basic Example Code
 * by: Kris Winer
 * date: April 1, 2014
 * license: Beerware - Use this code however you'd like. If you
 * find it useful you can buy me a beer some time.
\end{verbatim}

Pats programinis kodas yra parašytas labai netvarkingai ir yra požymių, kad šis programinis kodas buvo panaudotas iš kito projekto arba kitos įrangos iteracijos, kadangi pačiam projekte yra bibliotekų, kurie neturi visiškai jokios naudojamos srities šiame sprendime. Kaip pavyzdys gali būti LCD ekranas, kurio biblioteka yra įkeliama, tačiau niekur nepanaudota.

Kaip labai panašų sprendimą, kuris mūsų platformai nėra tinkamas, yra \cite{MPU6036:online}. Kontrolinis kodas labai panašus į \textit{MPU 9250} atvejį, skiriasi tik pačio bendravimo su senesniu jutikliu \textit{MPU 6036} programinis kodas. Į šią kategoriją taip pat papuola ir \cite{MPU9152:online}, šiuo atveju jutiklis yra \textit{MPU 9152}

Panaudota licencija yra suderinama su BSD licencija, todėl rastas programinis sprendimas galimas panaudoti ir esamai sistemai.

Reikia paminėti, jog panaudojamas programinis kodas parašytas labai chaotiškai, ir kadangi licencijos reikalavimai neriboja darbo su programine įranga, pats sprendimas yra perrašytas su geresniu pateikimu.