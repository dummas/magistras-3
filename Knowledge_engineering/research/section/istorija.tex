Duomenų gavybos ištakas galima paaiškinti kaip natūralią informacinių technologijų evoliuciją. 
Duomenų bazės ir duomenų valdymo industrija evoliucionavo iš poros kritinių funkcionalumų vystymo: duomenų surinkimas ir duomenų bazės sukūrimas, duomenų valdymas ir pažangus duomenų analizavimas. 
Ankstyvi duomenų surinkimo ir duomenų bazių kūrimo mechanizmai veikę kaip pagrindas tolimesniam efektyvesnių mechanizmų vystymui duomenų skaitymui ir rašymui, kaip ir užklausos ir transakcijos apdorojimui.
Dabartinės duomenų bazių sistemos jau numatytai suteikia užklausos ir transakcijos apdorojimą.
Pažangus duomenų analizavimas natūraliai tampa sekančiu žingsniu. 

Nuo 1960 metų, duomenų bazės ir informacinės technologijos evoliucionavo sistematiškai nuo primityvių bylų apdorojimo sistemų iki rafinuotos ir galingos duomenų bazės sistemos.
Tyrimas ir duomenų bazių įgyvendinimas nuo 1970 metų pažengė nuo hierarchinių ir tinklo duomenų bazių sistemų iki reliacinės sistemos, duomenų modeliavimo įrankių, indeksavimo ir priėjimo metodų.
Taip pat, naudotojai gavo labai praktinį ir lankstų priėjimą prie duomenų per užklausos kalbas, naudotojo sąsajas, užklausos optimizavimą ir transakcijos valdymą.
Efektyvūs metodai tinklo transakcijos apdorojimui, kur į užklausą yra žvelgiama kaip į skaitymo transakcija, suvaidino svarbų vaidmenį plačioje reliacinės technologijos adaptacijoje, kaip pagrindinį įrankį efektyviam duomenų saugojime, skaityme ir didelių duomenų kiekio valdyme.

Pažangios duomenų analizės sistemų vystymas prasidėjo nuo  1980 metų.
Stabilus ir neįtikėtinas kompiuterinės įrangos progresas per ankstesnius tris dešimtmečius suteikė pažangių ir pigių kompiuterių -- duomenų surinkimo, saugojimo įrenginių.
Tokia technologija leido labai pasitempti duomenų bazių ir informacinių technologijų industrijai. Ji leido didelius kiekius duomenų bazių ir informacijos saugyklų būti pasiekiamai per transakcija, informacijos nuskaitymą, bei duomenų analizę.
Dabar duomenys gali būti saugojami skirtingose duomenų bazių saugyklose.

Viena sparčiausiai augančių duomenų saugojimo architektūrų yra \textit{data warehouse}.
Tai yra didelė struktūra iš kelių skirtingų duomenų šaltinių, organizuotų po viena duomenų schema vienoje vietoje, kuri įgalina priiminėti sprendimus.
Duomenų centro technologija susideda iš duomenų valymo, duomenų integravimo ir analitinio tinklo apdorojimo -- analizavimo technikos, iš išvadų sudarymo, įtvirtinimo ir agregavimo. Taip pat galimybė peržiūrėti informaciją iš skirtingų kampų.
Nors šie įrankiai leidžia daugybės dimensijų analizavimą ir priimti kažkokį sprendimą, papildomi analizės įrankiai yra reikalingi gilesnei analizei, kaip pavyzdžiui duomenų gavybos įrankiai, kurie suteikia galimybe klasifikavimui, grupavimui, neįprastiems požymiams rasti ir pokyčių ieškojimui per laiko tarpą.

Dideli kiekiai duomenų taip pat yra surenkami už duomenų bazių ir saugojimo stočių. Per 1990 atsirado bendras interneto tinklas ir su juo pradėjo labai sparčiai plisti internetinės duomenų bazės, kaip pavyzdžiui yra \textit{XML}.
Tinklu paremtos informacijos bazės iškilo ir turi labai svarbią sritį informacinėje srityje.
Efektyvi ir naudinga tokių duomenų formos analizė yra labai didelis ir sunkus uždavinys.

Susidaro labai didelė problema -- didelis skaičius duomenų nesąlygoja visiškai kažkokių žinių, kurias suteikia toks jų skaičius.
Limitas, kuris nusako vieno žmogaus sugebėjimus analizuoti duomenis ir išgauti iš jų kažkokia informacija rankiniu būdu jau seniai yra peržengta.
Tokiu būdu duomenų saugyklos tapo tiesiog duomenų kapinėmis, kurias niekas neanalizuoja.
Tokiu būdu kritiniai sprendimai yra dažniausiai priimami ne atsižvelgiant į informaciją, kuri gali būti išgauta iš turimų surinktų duomenų, o į priimančios pusės intuicijos.
Buvo atlikti bandymai rankiniu būdu integruoti duomenys ir srities žinias priiminėti tam tikrus sprendimus, tačiau toks bandymas nepavyko, kadangi tokios sistemos dažniausiai yra stipriai veikiamos žmonių nešališkumo.
Labai reikia sistemiškai atlikinėti žinių išgavimo įrankių vystymą, tuomet esami duomenų kapai taps auksiniais kiaušiniais.



