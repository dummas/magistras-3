Žinių išgavimas yra procesas, kurio tikslas yra išgauti įdomius duomenų modelius iš didelio kiekio duomenų.
Tai taip pat yra vadinama žinių atradimo procesu ir jis tipiškai susideda iš duomenų valymo, integravimo, išrinkimo, transformavimo, modelio sukūrimo, jo įvertinimo ir gautų žinių atvaizdavimo.

Duomenų modelis yra įdomus jeigu jis galioja panaudojus testinius duomenis su tam tikru pasikliovimo lygiu, neįprastas ir potencialiai yra pritaikomas, paprastas suprasti žmogui.
Įdomūs duomenų modeliai sudaro žinias. 

Žinių inžinerija yra daugelio dimensijų požiūris į duomenis. Pagrindinės dimensijos yra duomenys, žinios, technologijos ir programos.

Žinių išgavimas gali būti suburtas iš skirtingo tipo duomenų, tol ko duomenis turi prasmę norimai sričiai, kaip duomenų bazės duomenis, duomenų saugyklos duomenis, perdavimo duomenis, ir pažengę duomenų tipai. 
Pažengę duomenų tipai susideda iš laikinės ar sekos duomenis, duomenų srautas, tūrinė informacija, tekstas ir daugialypės terpės duomenis, grafų ir tinklo duomenis.

Duomenų saugykla yra vieta saugoti duomenis ilgam laikotarpiui iš skirtingų šaltinų, bendroje schemoje.
Saugyklos sistemos pateikia daugelio dimensijų duomenis su analizės palaikymu, taip pat vadinamai realaus laiko analizės apdorojimui.

Daugelio dimensijų žinių išgavimas, integruoja pagrindinius žinių išgavimo metodus su realaus laiko daugelio dimensijų analize.
Metodas ieško įdomių modelių per skirtingas ir įvairias duomenų kombinacijas per skirtingas dimensijas, taikant skirtingas abstrakcijas, taip labai giliai analizuojant sąryšius tarp skirtingų dimensijų.

Žinių išgavimo funkcionalumai yra naudojami aprašant modelio arba žinių tipus, kurie buvo rasti per žinių išgavimą.
Tokie funkcionalumai yra charakterizavimas ir diskriminavimas, dažninių modelių išgavimas, asociacija ir koreliacija, klasifikavimas ir regresija, grupinė analizė; išskirčių aptikimas.

Žinių išgavimas turi labai daug pasisekusių įgyvendinimų, verslo srityje, interneto paieškoje, biologijos, sveikatos informatikoje, finansų, bibliotekos ir vyriausybės srityje.