Žinių išgavimas yra procesas, kurio tikslas yra išgauti įdomius duomenų modelius iš didelio kiekio duomenų.
Tai taip pat yra vadinama žinių atradimo procesu ir jis tipiškai susideda iš duomenų valymo, integravimo, išrinkimo, transformavimo, modelio sukūrimo, jo įvertinimo ir gautų žinių atvaizdavimo.

Duomenų modelis yra įdomus jeigu jis galioja panaudojus testinius duomenis su tam tikru pasikliovimo lygiu, neįprastas ir potencialiai yra pritaikomas, paprastas suprasti žmogui.
Įdomūs duomenų modeliai sudaro žinias. 
