Prieš tai esančiame skyriuje buvo aprašyti skirtingi duomenų ir informacijos šaltinių saugyklos, kurias naudojant galima atlikti žinių išgavimą.
Dabar reikia panagrinėti kokie duomenų modeliai gali būti išgauti.

Egzistuoja pora žinių išgavimo funkcionalumai. 
Į juos įeina charakterizavimas ir diskriminavimas; išgavimas dažninių modelių, asociacijos ir koreliacijos; klasifikavimas ir regresija; grupės analizė; išskirčių analizė.
Žinių išgavimo funkcionalumai yra naudojami specifikuojant tipus modelių, kurie gali būti rasti atliekant žinių išgavimo operaciją.
Bendru atveju, tokie uždaviniai gali būti suskirstyti į dvi kategorijas: apibūdinantys ir nuspėjantys.
Apibūdinantis žinių išgavimas charakterizuoja duomenų savybes į adresuojamą duomenų rinkinį.
Nuspėjantis išgavimas atlieka esamų duomenų indukciją, kad atlikti spėjimus.

Žinių išgavimo funkcionalumai, ir kiti galimi atradimui modeliai, yra aprašomi žemiau. Įdomūs duomenų modeliai sudaro žinias.

\subsection{Charakterizavimas ir diskriminavimas}

Duomenų esybės gali būti surištos su objektu ar koncepcija.
Pavyzdžiui, \textit{AllElectronics} parduotuvėje, prekių pardavimui klasės 

\subsection{Dažniniai modeliai, asociacijos ir koreliacijos}

\subsection{Klasifikavimas ir regresija}

\subsection{Grupinė analizė}

\subsection{Išskirčių analizė}

\subsection{Modelių įdomumas}