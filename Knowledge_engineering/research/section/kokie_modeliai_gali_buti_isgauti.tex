Prieš tai esančiame skyriuje buvo aprašyti skirtingi duomenų ir informacijos šaltinių saugyklos, kurias naudojant galima atlikti žinių išgavimą.
Dabar reikia panagrinėti kokie duomenų modeliai gali būti išgauti.

Egzistuoja pora žinių išgavimo funkcionalumai. 
Į juos įeina charakterizavimas ir diskriminavimas; išgavimas dažninių modelių, asociacijos ir koreliacijos; klasifikavimas ir regresija; grupės analizė; išskirčių analizė.
Žinių išgavimo funkcionalumai yra naudojami specifikuojant tipus modelių, kurie gali būti rasti atliekant žinių išgavimo operaciją.
Bendru atveju, tokie uždaviniai gali būti suskirstyti į dvi kategorijas: apibūdinantys ir nuspėjantys.
Apibūdinantis žinių išgavimas charakterizuoja duomenų savybes į adresuojamą duomenų rinkinį.
Nuspėjantis išgavimas atlieka esamų duomenų indukciją, kad atlikti spėjimus.

Žinių išgavimo funkcionalumai, ir kiti galimi atradimui modeliai, yra aprašomi žemiau. Įdomūs duomenų modeliai sudaro žinias.

\subsection{Charakterizavimas ir diskriminavimas}

Duomenų esybės gali būti surištos su objektu ar koncepcija.
Pavyzdžiui, \textit{AllElectronics} parduotuvėje, prekės, kurios yra parduodamos, yra suskirstytos į dvi kategorijas -- kompiuteriai ir spausdintuvai. Parduotuvės pirkėjai gali būti skirstomi į biudžetinius ir daug išleidžiančius pirkėjus.
Yra labai naudinga aprašyti esamas klases.
Toks aprašymas arba sąvoka yra vadinama klasės aprašymas.
Aprašymai gali būti išvesti iš duomenų charakterizavimo, grupuojant duomenis pagal tyrimą bendromis sąvokomis arba diskriminuojant, lyginant planinę klasę su viena ar rinkiniu palyginamų klasių arba taikant abu metodus.

Duomenų charakterizavimas yra grupavimas pagal pagrindinius planinės klasės charakteristikas arba ypatybes. 
Duomenis, kurie priklauso tam tikrai klasei, yra dažniausiai pateikiami naudotojo taikant užklausą.
Pavyzdžiui, norint išsiaiškinti kokie programinės įrangos paketo pardavimai praeitais metais padidėjo 10\%, priklausantys duomenis gali būti pasiekiami paleidžiant užklausą į duomenų bazę.
Egzistuoja keli efektyvūs duomenų grupavimo ir charakterizavimo metodai.
Paprasčiausias duomenų grupavimas yra paremtas statistiniais įverčiais.

Duomenų charakterizavimo išvados gali būti pateikiamas skirtingomis formomis.
Pavyzdžiui pyrago, juostine diagrama, kreive, daugelio dimensijų duomenų kubais ir daugelio dimensijų lentelėmis.
Gauti aprašymai gali būti atvaizduoti kaip bendri sąryšiai ar taisyklės pavidalu, kitaip vadinai charakteristikos taisyklėmis.

Duomenų diskriminavimas yra planinės klasės duomenų pagrindinių ypatybių palyginimas su kitų objektų ypatybėmis.
Planinės ir lyginamosios klasės gali būti pateiktos naudotojo, o reikiami duomenų objektai gali būti pateikiami užklausos pagalba.
Gaunamas rezultatas yra labai panašus į duomenų charakterizavimo rezultatą, tačiau diskriminavimo aprašymas turi įtraukti palyginamuosius įverčius, kurie leidžia atskirti planines ir kontrastines klases.
Diskriminuoti aprašymai yra pateikiami diskriminanto taisyklių forma.

\subsection{Dažniniai modeliai, asociacijos ir koreliacijos}

Dažniniai modeliai, kaip sąlygoja pavadinimas, yra modeliai, kurie pažymi duomenų dažnumą.
Egzistuoja daug dažninių modelių, tarp jų -- objektų grupių dažniai, objektų eilės dažniai, dažnių struktūros. 
Grupiniai dažniai dažniausiai yra aprašomi objektai, kurie pasitaiko grupėje labai dažnai.
Pavyzdžiui pienas ir duona, kurie dažniausiai yra perkami kartu maisto prekių parduotuvėje.
Eilės dažnis yra toks objektų dažnis, kurie seka vienas po kito labai dažnai.
Pavyzdžiui pirkėjas pradžioje perka nešiojama kompiuterį, paskui skaitmeninę kamerą ir tuomet atminties kortelę.
Dažninės struktūros pažymį struktūrines formatas, kurios gali būti sujungiamos su grupiniais ir eilės dažnių modeliais. 
Kaip pavyzdys gali būti grafas, medis, medžio lapas.
Žinių išgavimas dažniais modeliais veda prie įdomių asociacijų ir koreliacijų atskleidimo duomenyse.

\subsection{Klasifikavimas ir regresija}

Klasifikavimas yra procesas, kurio tikslas yra rasti modelį, arba funkcija, kuri geriausia apibrėžia ir atskiria duomenų klases arba jų sąvokas.
Modelis yra kuriamas testuojamų duomenų pagrindu. 
Modelis yra naudojamas nuspėti klasės etiketėmis objektams, kurie neturi jokios etiketės.

Gautas modelis gali būti pateiktas skirtingomis formomis -- kaip pavyzdžiui klasifikavimo taisyklėmis, sprendimo medžiais, matematine formuluote ar dirbtinių neuronų tinklais.
Sprendimo medis yra struktūra, panaši į struktūrinę schemą, kur kiekvienas mazgas atlieka sprendimą pagal kažkokios ypatybės atributo vertę, kiekviena šaka atspindi galima loginės operacijos rezultatą, lapeliai atspindi turimas objektų grupes arba klases.
Sprendimo medis gali lengvai būti koncertuojamas į klasifikavimo taisykles.
Dirbtinių neuronų tinklas, kuomet naudojamas klasifikavimui, dažniausiai susideda iš panašaus į neuroną apdorojimo mazgą, su įvertintais sujungimais tarp vienetų.
Egzistuoja daug kitų būtų konstruoti klasifikavimo modelį, kaip pavyzdžiui naivusis Bayesian klasifikatorius, vektoriaus palaikymo mašina ir k-artimi-kaimynai klasifikavimas.

Kur klasifikavimas apsprendžia kategorinius žymeklius, regresiniai modelių pagalba galima atlikti nepertraukiamą funkcijos įverčių skaičiavimą.
Tai reiškia, kad regresija gali būti naudojama nuspėti neturimus duomenų įverčius, o ne žymeklius.
Nuspėjimo sąvoką galioja tiek skaitinei, tiek žymėjimo išraiškai.
Regresinė analizė yra statistinė metodologija, kurie dažniausia yra naudojama skaitiniam nuspėjimui. 
Verta paminėti, jog egzistuoja ir kiti metodai. 
Regresija taip pat apima pasiskirstymo ir tendencijų identifikavimą, turint esamus duomenis.

Klasifikavimas ir regresija gali būti panaudojami prieš aktualumo analizę, kurios tinklas yra atpažinti atributus, kurie stipriai įtakoje klasifikavimą ir regresijos procesą.
Tokie atributai gali būti panaudoti klasifikavimo ir regresijos metu.
Kiti atributai, kurie neturi įtakos, gali būti išmesti iš svarstymo.

\subsection{Grupinė analizė}

Skirtingai nuo klasifikavimo ir regresijos, kurios analizuoja pažymėtus duomenis, grupinė analizė dirba su duomenimis, kurie nėra pažymėti.
Daugelio atveju duomenų apdorojimo pradžioje, gali neegzistuoti pažymėtų duomenų.
Tokiu atveju, grupinė analizė gali būti panaudota priskiriant pradinius žymeklius duomenų grupėmis.
Objektai yra grupuojami pagal ``vidinės klasės panašumo maksimumą ir išorinės klasės panašumo minimumą'' principą.
Tai reiškia, kad objektų grupės yra formuojamos tokiu principu, jog objektai grupės viduje turi daug panašumu vienas su kitu ir labai mažai panašumų su objektais už grupės ribų. 
Kiekviena grupė yra formuojama tokiu būdu, jog kiekvienai grupei galima sudaryti taisyklių sąrašą.
Grupės sudarymas gali palengvinti taksonomijos formavimą -- sąvokų hierarchijų stebėjimą pagal grupuotus įvykius.

\subsection{Išskirčių analizė}

Duomenų rinkinys gali turėti objektų, kurie tiesiog nėra surišami su likusiais objektais pagal elgseną ar modelį. 
Tokie duomenys vadinami išskirtys.
Daugelis žinių išgavimo metodai išmeta tokius duomenis kaip triukšmas ar išskirtinumai.
Tačiau, kai kurios taikytinos sritys (sukčiavimo atpažinimas) išskirtiniai atvejai yra žymiai įdomesni už paprastuosius.
Tokia analizė vadinama išskirčių analizė arba anomalijų išgavimas.

Išskirtys gali būti atpažintos naudojant statistinius bandymus, kurie priima pasiskirstymą arba duomenų tikimybės modelį arba naudoja nuotolio matavimus tarp objektų, kurie yra nutolę toliau negu kito objektai grupėse.

\subsection{Modelių įdomumas}

Žinių išgavimo sistemos turi potencialą generuoti milijonus tūkstančių modelių bei taisyklių.
Tačiau kyla klausimas ar visi modeliai yra įdomus? Greitas ir paprastas atsakymas yra -- ne.
Tik maža dalis modelių potencialiai gali generuoti kažkokį dėmesį duotam naudotojui.

Tokia problema kelia rimtą klausimą žinių išgavimo sistemoms -- kaip galima įvertinti gautą modelį? Kaip įvertinti, kad jis yra įdomus ir turi kažkokios naudos naudotojui?

Modelis yra įdomus tokiais atvejais, kai jis yra paprastai suprantamas naudotojui, galioja su tam tikru užtikrintumo lygiu duotiems duomenims, potencialiai yra naudingas bei išskirtinis.
Modelis taip pat yra įdomus, jeigu jis patvirtinta hipotezę, kurią naudotojas turėjo iškėlęs prieš atliekant tyrimą. 
Įdomūs modeliai pat pat yra vadinami žiniomis.

Egzistuoja pora objektyvių vertinimo mechanizmų.
Jie visi yra parmeti modelių radimo metodika ir po ja esančia statistika.
Objektyvus taisyklės asociacijos matas turi forma $ X \rightarrow Y$, kuri vaizduoja procentine išraiška apie transakcija pagal esamas transakcijos taisykles.
Toliau tai galima išreikšti tikimybe $P(X \cup Y)$, kur $X \cup Y$ žymi, kad transakcija egzistuoja $X$ ir $Y$ junginyje.
Kitas objektyvus matas yra patikimumo matas, kuris suriša rastos taisyklės užtikrintumą.
Tai galima išreikšti sąlygine tikimybe $P(Y|X)$ -- tikimybe, kad transakcija, kuris susideda iš $X$, taip pat susideda ir iš $Y$.

