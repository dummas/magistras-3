Žinių išgavimas gali būti apibūdinamas labai skirtingai, kadangi šitas mokslas priklauso daugeliui sričių.
Netgi terminas \textit{žinių išgavimas}, nesudaro viso bendro vaizdo.
Terminas turėtų būti labiau susijęs su sritimi, kaip pavyzdžiui ``žinių išgavimas iš duomenų'', tačiau jis yra labai ilgas.

Daugelis žmonių žinių išgavimą laiko sinonimu su ``žinių atradimui iš duomenų'', tuo tarpu kiti į terminą žiūri tiesiog kaip svarbų žingsnį žinių atradimo procese. Žinių atradimo procesas susideda iš žingsnių:

\begin{itemize}
    \item Duomenų valymas. Triukšmo ir netolydžių duomenų šalinimas.
    \item Duomenų integravimas. Jeigu yra keli duomenų šaltiniai ir juos reikia sujungti.
    \item Duomenų pasirinkimas. Duomenys, kurie turi svarbą nagrinėjamam atvejui yra nuskaitomi iš duomenų bazės.
    \item Duomenų transformavimas. Duomenys yra transformuojami ir sudaromi į formas, kurios yra tinkamos išgavimui, ir yra galimos bendrinimo ir grupavimo operacijos.
    \item Duomenų išgavimas. Svarbus žingsnis, kuriuo metu yra pritaikomi intelektualūs įrankiai išskirti modeliui.
    \item Modelio įvertinimas. 
    \item Žinių pateikimas. Vizualus rasto modelio pateikimas naudotojui.
\end{itemize}

Nurodyti žingsniai parodo, kad žinių išgavimas yra žinių atradimo proceso dalis, tiksliau galima pasakyti, kad šitas žingsnis yra esminis, kadangi jis atskleidžia paslėptus modelius, kuriuos galima vertinti.
Reikia pastebėti, jog industrijoje, žiniasklaidoje ir tyrimuose, žinių išgavimo sąvoka naudojama išreikšti visą žinių atradimo procesą, dėl trumpumo.
Dėl šios priežasties, žinių išgavimo funkcionalumas yra apibūdinamas kaip įdomių modelių ir žinių atradimas iš didelio skaičiaus duomenų.
Duomenų šaltinis gali būti duomenų bazės, saugyklos, internetas ir kitos informacijos talpyklos, arba duomenys, kurie yra gaunami srautu, dinamiškai.