Multimedijos turinio filtravimas

Filtravimo žingsniai

\begin{itemize}
    \item Segmentavimas;
    \item Klasifikavimas;
    \item Sekimas;
    \item Atpažinimas;
\end{itemize}

%% Multimedijos turinio filravimas yra panaudojimas vaizdo turinio atpažinimui, kuris yra reikalingas nenorimo media turinio pašalinimui, autorinių teisių išsaugojimui. Filtravimas susideda iš kelių žingsnių. Pirmas žingsnis yra vaizdo vaizdo turinio segmentavimas. Tai yra procesas, kurio tikslas yra išskaidyti vaizdą į jo sudedamąsias dalis -- atskirti foną nuo priekinio vaizdo ir sudaryti savybių erdvę. Antas žingsni yra klasifikavimas, kuomet yra dirbama tiesiogiai su savybių erdve. Tolimesnis veiksmas yra tų pačių klasifikuotų savybių sekimas per kadrų seką. Taip iš vieno ir to paties klasifikuoto objekto judėjimo tendencijos galime atpažinti objektą.