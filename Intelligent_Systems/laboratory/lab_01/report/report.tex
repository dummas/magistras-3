\documentclass[11pt, a4paper, lithuanian]{article}

\usepackage[left=25mm,right=15mm,top=15mm,bottom=15mm]{geometry}
\usepackage[utf8x]{inputenc}
\usepackage[L7x]{fontenc}
\usepackage[lithuanian]{babel}
\usepackage{listings}
\usepackage{amsmath, amssymb}

\author{AKSfm-15, Maksim Norkin}
\title{Perceptrono mokymas}

\lstset{
  language=Matlab,
  basicstyle=\footnotesize,
  columns=fixed,
  numbers=none,
  showspaces=false,
  xleftmargin=20pt
}

\begin{document}

    \maketitle

    \section{Užduotis}


        Nenaudojant specializuotų matlab funkcijų, sukurti perceptroną-klasifikatorių ir jį apmokinti obuolių ir kriaušių nuotraukomis klasifikuoti; Galimi požymiai, apskaičiuojami iš nuotraukų yra spalva ir forma;

        Perceptrono apmokymui panaudoti 5 nuotraukas: 2 iš kurių yra kriaušės, 5 obuoliai; Apmokytą perceptroną patikrinti, naudojantis likusias obuolių ir kriaušių nuotraukas;

    \section{Pradinės sąlygos}

        Naudojamas dviejų įėjimų perceptronas, todėl mums reikia dviejų sverties kintamųjų ir vienos bazės; Esant didesniam perceptronų kiekui, pradinių bazių skaičius būtų lygūs perceptronų skaičiui, kadangi kiekvienas perceptronas turi turėti savo bazę;

        Perceptroną parenkame paprastą, su \textit{sign} tipo perdavimo funkcija; Taip perceptronas matematine formuluote turi atrodyti taip:

        \begin{equation}
            y = \sum_{n} \sigma(w_n*x_n + b),
        \end{equation}
        kur $n$ -- skaičius įėjimų į perceptroną, o $\sigma$ -- perceptrono slenkties funkcija;

        Slenkties funkcija parenkame paprastą $sign$, kadangi mes turime binarinį išėjimą, t.y. išėjime galimi tik du klasifikavimo variantai; o $n$ yra 2, kadangi turime tik du įėjimus; Taip sudaroma tokia fomuluotė:

        \begin{equation}
            y = sign(w_1 * x_1 + w_2 * x_2 + b);
        \end{equation}

        Pradinės sąlygos perceptronui, yra parenkamos atsitiktinai, kaip yra pateikta \ref{code:pradines_salygos} pavyzdyje; 

        \begin{figure}[h]
            \centering
            \caption{Pradinės sąlygos}
            \label{code:pradines_salygos}
            \lstinputlisting{sources/pradines_salygos.m}
        \end{figure}

    \section{Mokymas}

        Perceptrono mokymas įgyvendinamas, panaudojus mažiausio vidutinio kvadrato metodą, kuris matematikai atrodo taip:

        \begin{equation}
            \mathbf{\hat{h}}(n+1) = \mathbf{\hat{h}}(n) + \mu e^*(n)\mathbf{x}(n)
        \end{equation}

        Skaičiavimus pradedame nuo pirminio atsakymo skaičiavimo, naudojant pradines sąlygas, nustatytas ankščiau;

        \begin{equation}
            \mathbf{Y} = sign(\mathbf{X}'\mathbf{w} + \mathbf{b}),
        \end{equation}
        patikriname esamą rezultatą su rezultato vektoriumi, kurį tikimės gauti $\mathbf{T}$
        \begin{equation}
            E_{err} = \sum \mathbf{T} - \mathbf{Y},
        \end{equation}
        taip gauname kokiu mastu gautas rezultatas buvo klaidingas; Gautas įvertis toliau yra naudojamas kaip klaidos įvertis atnaujinant sistemos kintamuosius, $\mathbf{w}$ ir $b$. Įverčių atnaujinimas vykdomas sekančia išraiška, remiantis mažiausio vidurkio kvadrato metodu:
        \begin{equation}
            \mathbf{w} = \mathbf{w} - \mu*E_{err}*\sum\mathbf{X},
        \end{equation}
        kur $\mathbf{X}$ -- įėjimo verčių vektorius, ir $\mu$ yra apmokymo žingsnio koeficientas; Bazė atnaujinama, nepriklausomai nuo įverčių vektoriaus
        \begin{equation}
            b = v - \mu*E_{err}
        \end{equation}

        Programiniu kodu visas aprašytas matematinis modelis yra aprašomas \ref{code:mokymas} pavyzdyje;

        \begin{figure}[h]
            \centering
            \caption{Mokymas}
            \label{code:mokymas}
            \lstinputlisting{sources/klaidos_skaiciavimas.m}
        \end{figure}

    \section{Klasifikatoriaus rezultatas}

        Klasifikatoriaus rezultatas yra apskaičiuojamas lygiai taip pat, kaip ir vyko mokymas. Skirtumas nuo mokymo -- nėra atnaujinami modelio parametrai;

        Rezultatas buvo patikrinamas su 8 likusiais paveikslėliais, iš kurių neteisingai buvo klasifikuoti 3.

    \section{Išvados}

        Laboratorinio darbo metu, buvo sukurtas perceptrono klasifikatorius. Jis buvo mokintas obuolių ir kriaušių nuotraukomis klasifikuoti; Kaip požymiai buvo panaudoti nuotraukos spalva ir forma, esanti nuotraukoje;

        Perceptrono apmokymui buvo panaudotos penkios nuotraukos, algoritmo veikimas patikrintas su likusiomis nuotraukomis

\end{document}
